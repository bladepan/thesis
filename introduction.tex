\section{Introduction}
\label{sec:intro}

% backgroud, motivation, design choices, architecture , experiments(goals, etc.)
Before AJAX~\cite{garrett2005ajax} became pervasive, most web site works like this:
the user sends a request to the web server through submitting a form or clicking a link on a web page,
then the server processes this request and responds a new HTML document.
In this model, the application logic resides mainly on server side, the client side is mostly
plain HTML document.
The problem about this form/link based model is that it is hard
to create responsive and rich user experience because the whole user interface
is wiped out and re-rendered every time user sends a request.

AJAX is an approach that uses javascript to send requests to server
and partially update the HTML document without page refresh.
It is capable of delivering a better performance and native application like user experience.
In this model, developers have to write client javascript code to handle server client communication
and rendering logic. 


\cb{} is a server-centric framework designed to simplify the development of AJAX web applications.
Developer's code is running in a server side virtual browser and the user's browser is just
a dumb display device which synchronizes with the virtual browser.
In \cb{}, developers use nothing but HTML, css and javascript to construct the UI logic just like
any traditional AJAX application.
In the place where traditional AJAX applications call a server side API through HTTP,
developers could call a server side method directly.
The synchronization between the virtual browser and the actual browser is 
handled by the framework under the hood.
\cb{} also naturally preserves UI state upon page refreshes
because all the UI state is kept in the server side.



%Comparing to other server-centric frameworks, 
%\cb{} could reuse most of existing client code because it does not require an extra markup language
%and its sole programming language is javascript.


The original \cb{}'s architecture is restricted to a single process.
It cannot benefit from multiple processors and provides no isolation between virtual browsers.
The stateful nature of \cb{} makes it hard to support multiple process, 
but we think it is a effort worth taking for it would not only boost \cb{}'s capacity
but also shed some light on how to scale nodejs applications in general.




