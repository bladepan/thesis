%
% PROJECT: <ETD> Electronic Thesis and Dissertation Initiative
%   TITLE: LaTeX report template for ETDs in LaTeX
%  AUTHOR: Neill Kipp, nkipp@vt.edu
%     URL: http://etd.vt.edu/latex/
% SAVE AS: etd.tex
% REVISED: September 6, 1997 [GMc 8/30/10]
% 

% Instructions: Remove the data from this document and replace it with your own,
% keeping the style and formatting information intact.  More instructions
% appear on the Web site listed above.

\documentclass[12pt]{report}

\usepackage{epsfig,endnotes}

\usepackage{color}
\usepackage{epstopdf}
\usepackage{psfrag}
\usepackage{subfigure}
\usepackage{xspace}
\usepackage{booktabs}
%\usepackage{float}
\usepackage[hyphens]{url}
\usepackage{path}
\usepackage[font=small,labelfont=bf]{caption}
\usepackage{courier}
\usepackage{listings}
\usepackage{pifont}
\usepackage{graphicx}
\usepackage{hyperref}
\usepackage{minted}
\usepackage{multirow}


\setlength{\textwidth}{6.5in}
\setlength{\textheight}{8.5in}
\setlength{\evensidemargin}{0in}
\setlength{\oddsidemargin}{0in}
\setlength{\topmargin}{0in}

\setlength{\parindent}{0pt}
\setlength{\parskip}{0.1in}

% Uncomment for double-spaced document.
\renewcommand{\baselinestretch}{1.5}

% \usepackage{epsf}

\begin{document}
\newcommand{\cb}{Cloud\-Browser\xspace}
\newcommand{\projectname}{Cloud\-Browser\xspace}
\newcommand{\cbtwo}{Cloud\-Browser 2.0\xspace}
\newcommand{\js}{Java\-Script\xspace}
\newcommand{\nodejs}{Node.js\xspace}
\newcommand{\appins}{App Instance\xspace}
\newcommand{\jsdom}{JSDOM\xspace}
\newcommand{\citemain}{~\cite{mcdaniel2012cloudbrowser}}
\newcommand{\etdtitle}{Rich Cloud-based Web Applications with \cbtwo}

%% A ``long'' caption.  Long captions show up in the list of
%% tables/figures with only the first argument.  Both arguments
%% show up in the actual caption.
\newcommand{\longcaption}[2]{\caption[#1]{#1 #2}}

\newcommand{\webscaleoutfig}{
    \begin{figure}[tb]
    \centering
    \includegraphics[width=\textwidth]{../figs/web_scale_out}
    \caption{Scalable web server architecture}
    \label{fig:webscaleout}
    \end{figure}
}

\newcommand{\architectureoverview}{
    \begin{figure*}[ht]
    \centering
    \includegraphics[width=\textwidth]{../figs/architecture_overview}
    \caption{Single Process \cb{} Architecture Overview}
    \label{fig:cb1arch}
    \end{figure*}
}

\newcommand{\newarchitectureoverview}{
    \begin{figure*}[ht]
    \centering
    \includegraphics[width=\textwidth]{../figs/new_architecture_overview}
    \caption{Multiprocess Process \cb{} Architecture Overview}
    \label{fig:cb2arch}
    \end{figure*}
}


\newcommand{\appinstancefig}{
    \begin{figure}[ht]
    \centering
    \includegraphics[width=0.8\textwidth]{../figs/appInstance}
    \caption{Application Instance}
    \label{fig:appinstance}
    \end{figure}
}


\newcommand{\appbundlefig}{
    \begin{figure}[ht]
    \centering
    \includegraphics[width=\textwidth]{../figs/application_bundle}
    \caption{Folder structure of a \cb application}
    \label{fig:appbundle}
    \end{figure}
}

\newcommand{\chatappfig}{
    \begin{figure}[ht]
    \centering
    \includegraphics[width=0.8\textwidth]{../figs/chat_application}
    \caption[Sharing data among multiple virtual browser via application instance]
        {This figure shows how multiple virtual browsers can directly,
        and seamlessly share relevant application data, in this case chat messages,
        which then become part of the model that drives the presentation
        MVC framework.
    }
    \label{fig:chatapp}
    \end{figure}
}


\newcommand{\clickthroughput}{
    \begin{figure}[ht]
    \centering
    \includegraphics[width=\textwidth]{../gnuplot/click_throughput}
    \caption{Throughput of ``Back-to-back'' click application.}
    \label{fig:clickthroughput}
    \end{figure}
}


\newcommand{\clicklatency}{
    \begin{figure}[ht]
    \centering
    \includegraphics[width=\textwidth]{../gnuplot/click_latency}
    \caption{Latency of ``Back-to-back'' click application.}
    \label{fig:clicklatency}
    \end{figure}
}


\newcommand{\clickwaitthroughput}{
    \begin{figure}[ht]
    \centering
    \includegraphics[width=\textwidth]{../gnuplot/click_wait_throughput}
    \caption{Throughput of click application, after introducing artificial delay.}
    \label{fig:clickwaitthroughput}
    \end{figure}
}


\newcommand{\clickwaitlatency}{
    \begin{figure}[tb]
    \centering
    \includegraphics[width=\textwidth]{../gnuplot/click_wait_latency}
    \caption{Latency of click application, after introducing artificial delay.}
    \label{fig:clickwaitlatency}
    \end{figure}
}



\newcommand{\angularchatlatency}{
    \begin{figure}[tb]
    \centering
    \includegraphics[width=\textwidth]{../gnuplot/angularchat_latency}
    \caption{Latency of chat application with Angular.js.}
    \label{fig:angularchatlatency}
    \end{figure}
}


\newcommand{\jquerychatlatency}{
    \begin{figure}[tb]
    \centering
    \includegraphics[width=\textwidth]{../gnuplot/jquerychat_latency}
    \caption{Latency of chat application with JQuery.}
    \label{fig:jquerychatlatency}
    \end{figure}
}


\newcommand{\chatroomfig}{
    \begin{figure}[tb]
    \centering
    \includegraphics[width=\textwidth]{../figs/chatroom}
    \caption{Chat Room Application}
    \label{fig:chatroom}
    \end{figure}
}

\newcommand{\apphierarchyfig}{
    \begin{figure}[tb]
    \centering
    \includegraphics[width=0.8\textwidth]{../figs/application_hierarchy}
    \caption[Application deployment model]{Application deployment model: Hierarchy of applications, application instances, and virtual browsers.
    Note that a single virtual browser may be broadcast to multiple clients (cobrowsing).}
    \label{fig:appidhierarchy}
    \end{figure}
}

\newcommand{\memfig}{
\begin{figure*}[ht]
    \centering
    \includegraphics[width=\textwidth]{../gnuplot/resource_consumption}
    \caption[Resource Consumption of worker]{
    Resource Consumption of worker node running JQueryChat Application\\
X axis is time. Left Y axis corresponds to the red line of CPU usage.
Right Y axis corresponds to memory statistics.\\
After about 90s after the system boots up, the benchmark tool starts to simulate 
user workload.
When the benchmark tool sending requests, \emph{HeapUsed} fluctuates as the system creates new objects and garbage collector cleans dead objects.
When the \emph{HeapUsed} drops there is a steep surge of CPU usage, indicating garbage collector is working at that time.
    }
    \label{fig:mem}
\end{figure*}
}


\newcommand{\nodermifig}{
    \begin{figure}[ht]
    \centering
    \includegraphics[width=\textwidth]{../figs/nodermi}
    \caption[Overall Design of nodermi]{Overall Design of nodermi}
    \label{fig:nodermi}
    \end{figure}
}

% deprecated
\newcommand{\nodermimethodinvokefig}{
    \begin{figure}[ht]
    \centering
    \includegraphics[width=0.8\textwidth]{../figs/nodermi_method_invoke}
    \caption[Remote method invocation via a stub object]{Process A invoke a method of a stub object}
    \label{fig:nodermimethodinvoke}
    \end{figure}
}

% deprecated
\newcommand{\nodermicallbackfig}{
    \begin{figure}[ht]
    \centering
    \includegraphics[width=0.8\textwidth]{../figs/nodermi_callback}
    \caption[Remote callback via a stub function]{Process B invoke a callback that itself is a stub}
    \label{fig:nodermicallback}
    \end{figure}
}

\newcommand{\nodermiexamplefig}{
    \begin{figure}[tb]
    \centering
    \includegraphics[width=0.8\textwidth]{../figs/nodermi_example}
    \caption{Nodermi remote method invocation example}
    \label{fig:nodermiexample}
    \end{figure}
}

\newcommand{\nodermiobjmapfig}{
    \begin{figure}[tb]
    \centering
    \includegraphics[width=0.8\textwidth]{../figs/nodermi_objectmap}
    \caption[Nodermi memory management]{Nodermi memory management : Nodermi holds 
    strong references to local objects that are remotely referenced in \emph{object map},
    holds weak references to \emph{stub}s in \emph{stub map}.}
    \label{fig:nodermiobjmap}
    \end{figure}
}

%deprecated
\newcommand{\nodermiracefig}{
    \begin{figure}[ht]
    \centering
    \includegraphics[width=0.8\textwidth]{../figs/nodermi_race}
    \caption[Race condition when dereferencing a remote reference]
    {Race condition when dereferencing a remote reference, \emph{objB} is garbage collected
    when \emph{Process A} still has a stub referencing it.}
    \label{fig:nodermirace}
    \end{figure}
}


\newcommand{\nodermipassbyreffig}{
    \begin{figure}[tb]
    \centering
    \includegraphics[width=\textwidth]{../figs/nodermi_passbyreference}
    \caption[Nodermi passes arguments by reference]
    {Nodermi passes argument by reference: When passing an argument to a remote
    method call, a remote reference is created for the argument in the server
    process of the remote method. The exception is that 
    when the argument is a stub, nodermi creates
    remote reference for the argument's source object or directly locate
    the source object.}
    \label{fig:nodermipassbyref}
    \end{figure}
}

\newcommand{\nodrmipassbyvalfig}{
    \begin{figure}[tb]
    \centering
    \includegraphics[width=0.8\textwidth]{../figs/nodermi_passbyvalue}
    \caption[Nodermi passes arguments by value]
    {Nodermi passes argument by value: When passing an argument to a remote
    method call and the argument is a simple object with no methods
    , a new copy of the argument is created in the server
    process of the remote method. When the argument is of certain built-in types,
    a new copy is created via constructors, so the new copy has all the methods
    of the original arguments.}
    \label{fig:nodermipassbyval}
    \end{figure}
}

\newcommand{\apiclassfig}{
    \begin{figure}[ht]
    \centering
    \includegraphics[width=0.8\textwidth]{../figs/api_classes}
    \caption[API class design]{API class design: 
    The arrow points to the method's return value's type.
    For instance,
    The \emph{listApplications} returns
    a list of \emph{APIApplication} objects.
    ``:'' indicates an variable's class, ``currentApplication:APIApplication'' means
     \emph{currentApplication} is a \emph{APIApplication} object.
    }
    \label{fig:apiclass}
    \end{figure}
}


\newcommand{\apireferencefig}{
    \begin{figure}[ht]
    \centering
    \includegraphics[width=0.6\textwidth]{../figs/api_reference}
    \caption[API object structure]{API object structure : API objects only keep weak reference to internal objects}
    \label{fig:apireference}
    \end{figure}
}



% http://www.tablesgenerator.com/latex_tables
% supports load table in latex code
\newcommand{\instantiationStrategyTbl}{

    \begin{table}[htb]
    \centering

    \begin{tabular}{|c|l|l|l|l|}
    \hline
    \multirow{2}{*}{Instantiation Strategy} & \multicolumn{2}{l|}{Number of \appins{}s} & \multicolumn{2}{l|}{Number of Virtual Browsers} \\ \cline{2-5} 
     & Per App & Per User & Per Instance & Per Instance Per User \\ \hline
    multiInstance & more than 1 & more than 1 & more than 1 & more than 1 \\ \hline
    singleBrowserPerUser & more than 1 & more than 1 & more than 1 & 1 \\ \hline
    \multicolumn{1}{|l|}{singleInstancePerUser} & more than 1 & 1 & 1 & 1 \\ \hline
    \multicolumn{1}{|l|}{singleAppInstance} & 1 & NA & 1 & NA \\ \hline
    \end{tabular}

    \caption[Application Instantiation Strategy]{Application Instantiation Strategy : For \emph{singleAppInstance}
    all users share a single instance and virtual browser, so the columns for \emph{Per User} is ``NA''.}
    \label{tab:appinstantiationstrategy}
    \end{table}

}

\def\code#1{\texttt{#1}}
\def\nodermi{\texttt{nodermi\xspace}}

\thispagestyle{empty}
\pagenumbering{roman}
\begin{center}

% TITLE
{\Large 
\etdtitle{}
}

\vfill

Xiaozhong Pan

\vfill

Thesis submitted to the Faculty of the \\
Virginia Polytechnic Institute and State University \\
in partial fulfillment of the requirements for the degree of

\vfill

Master of Science \\
in \\
Computer Science and Applications

\vfill

Godmar V. Back, Chair \\
Eli Tilevich\\
Ali R. Butt

\vfill

April 28, 2015 \\
Blacksburg, Virginia

\vfill

Keywords: Web, Javascript, Node.js, Distributed System
\\
Copyright 2015, Xiaozhong Pan

\end{center}

\pagebreak

\thispagestyle{empty}
\begin{center}

{\large \etdtitle{}}

\vfill

Xiaozhong Pan

\vfill

(ABSTRACT)

\vfill

\end{center}

When developing web applications using traditional methods, the developers
need to partition the application logic into a client side and server side,
then implement these two parts separately (probably in two different
programming languages) and write the communication code to synchronize
application state between the two parts. \cb is a server-centric web
framework that eliminates this need of partitioning application entirely. In
\cb, the application code is executed in server side virtual browsers
which also preserves the application's presentation state. The client web
browsers act like rendering devices, they fetch and render representation
states from the virtual browsers. The client-server communication and user
interface rendering is implemented by the framework under the hood. The
applications are developed in a way similar to regular web pages, using no
more than HTML, CSS and \js. Since the user interface state is
preserved server side, the framework also provides a continuous experience for
users who can disconnect from the application at any time and reconnect to
pick up at where they left off.

The original implementation of \cb is single-threaded  and supported
deployment on only one process. We implemented \cbtwo, a multi-process
implementation of \cb. \cbtwo can be deployed on a cluster of servers as well
as a single multi-core server. It distributes the virtual browsers to multiple
processes and  dispatches client requests to the associated virtual browsers.
\cbtwo also refines the \cb application deployment model to make the framework
a PaaS platform. The developers can develop and deploy different types of
applications and the platform will automatically make them scalable.


% To transparently partition the existing CloudBrowser infrastructure
% code across nodes, we designed and implemented an object-oriented RPC
% framework called nodermi for node.js.  Nodermi transparently creates remote
% references during remote method invocations and garbage collects remote
% references when they are not needed. To better understand the limitations of
% the system and assess the feasibility of hosting real-life applications, we
% evaluated the system's performance using different types of applications and
% JavaScript libraries. Our experiments show that CloudBrowser 2.0 scales
% linearly, it can support 2,800 concurrent clients interacting with a non-
% trivial web application using a eight core machine.

% ------------

\vfill

% GRANT INFORMATION

\pagebreak

% Dedication and Acknowledgments are both optional
% \chapter*{Dedication}
% \chapter*{Acknowledgments}
\chapter*{Acknowledgments}
\markright{Acknowledgments}

I am very fortunate to have Dr. Godmar Back as my advisor.  I would like to
thank him for his patience and guidance. I am very grateful for the time and
effort he put in our weekly one-to-one meetings. These meetings are
at least equivalent to three graduate courses.

I would also like to thank Dr. Ali R. Butt and Dr. Eli Tilevich for serving
on my committee and providing valuable feedbacks.

I would also like to thank my fellow graduate students in computer science
department, it is a honor to spend two years with you wonderful guys.

Finally, I would like to thank National Science Foundation.
This thesis is based upon work supported by the National Science Foundation 
under Grant No. CCF-0845830.

\tableofcontents
\pagebreak

\listoffigures
\pagebreak

\listoftables
\pagebreak

\pagenumbering{arabic}
\pagestyle{myheadings}

\chapter{Introduction}
\markright{Introduction}

\section{Motivation}

Most newly developed applications that provide a user interface to end users 
are web-based.  Modern browsers provide powerful and expressive user interface 
elements, allowing for rich applications, and the use of a networked
platform simplifies the distribution of these applications.  As a result,
researchers and practitioners alike have devoted a great deal of attention to
how to architect frameworks on this platform, which is characterized by the
use of the stateless HTTP protocol to transfer HTML-based user interface 
descriptions to the client along with JavaScript code, which in turn 
implements interactivity and communication with the application's backend tiers.

In many recently developed frameworks, much of the presentation logic of these applications
executes within the client's browser.  User input triggers events
which result in information being sent to a server and subsequent user interface updates.
Such updates are implemented as modifications to an in-memory tree-based
representation of the UI (the so-called document object model, or DOM), which
is then rendered by the browser's layout and rendering engine so the user can see it.
However, the state of the DOM is ephemeral in this model: when a user visits 
the same application later from the same or another device, or simply reloads the page, the state of the
DOM must be recreated from scratch.  In most existing applications, this 
reconstruction is done in an rudimentary and incomplete way, because
application programmers typically store only application state, and little 
or no presentation state in a manner that persists across visits.
As a result, many web applications do not truly feel like persistent,
``in-cloud'' applications to which a user can connect and disconnect at will.
By contrast, users are accustomed to features such as Apple's Continuity\cite{apple-continuity}
that allows them to switch between devices while preserving not only 
essential data, but enough of the applications' view to create the appearance
of seamlessly picking up from where they left off.

\cb{}~\cite{mcdaniel2012cloudbrowser} is a server-centric web framework 
that keeps the state of the HTML document in memory 
on the server in a way that is persistent across visits.
In this model, presentation state is kept in virtual browsers whose life
cycle is decoupled from the user's connection state.  When a user is connected,
a client engine mirrors the state of the virtual browser in the actual browser which
renders the user interface the user is looking at.  Any events triggered by the
user are sent to the virtual browser, dispatched there, and any updates
are reflected in the client's mirror.  This idea is reminiscent of 
``thin client'' designs used in cloud-based virtual desktop offerings,
but with the key difference that in this proposed design the presentation state
that is kept in a virtual browser is restricted to what can be represented at the
abstract DOM level; no flow layout or rendering is performed by the virtual browser on the 
server.

This model entails additional potential benefits: since only framework code runs in
the client engine, the application code running on the server does not need
to handle any client/server communication and can be written in an event-based
style similar to that used by desktop user interface frameworks.
Since the virtual browser has the same JavaScript execution capabilities as a
standard browser, emerging model-view-controller (MVC) frameworks such as 
AngularJS~\cite{hevery2009angular} can be directly used, further simplifying application development.
More side benefits include: a lighter weight client engine that can load faster,
a resulting application that is potentially more secure since no direct access to
application data needs to be exposed, the ability to co-browse by broadcasting
the virtual browser state to multiple clients.


\section{Core Contributions}

Prior to this work, the implementation of \cb{} was single threaded and
supported deployment on only one process. It could not scale horizontally by
adding more processors or more machines to increase the system's capacity. To
enable \cb{} to host large scale web applications, we designed and implemented
\cbtwo which can distribute virtual browsers  across the CPU cores of a
multiprocessor machine or across a cluster of machines.

We designed a single-master, multiple-workers architecture:
the master is responsible for application management and load balancing,
the workers host virtual browsers and serve user requests.
To facilitate the inter-process
communication among the processes in our system, we developed a remote procedure call
framework nodermi~\cite{nodermi} that encapsulates message communication as
method calls.
Nodermi transparently creates stubs that represent objects in other processes
and allows a process to invoke methods of other processes' objects by calling
methods on the stubs.
We also developed an application programming interface that fully
isolates application code from each other and from underlying systems code.

The underlying architecture change is transparent to applications. The new
implementation fully preserves the semantics of the existing programming model
while provide higher scalability. \cb{} applications do not need to be aware of
the distributed architecture on which they run.  Most of the existing applications 
can automatically scale to multiple processors without modification. Some
applications needed to be modified because of necessary changes to API methods'
signatures.

We have implemented a number of sample applications and profiled them to
better understand the intrinsic and extrinsic limitations of this design.  We
also built a benchmark tool that simulates multiple users interacting with the
applications and used it to evaluate the performance and scalability of
\cbtwo. In our experiments,  \cbtwo scales linearly, it supports 2,800
concurrent users using a chat room application on a eight core machine with
average latency under 100ms.


\chapter{Background}
\markright{Background}

This chapter gives background information to understand the concept of \cbtwo.
We assume the reader has the basic knowledge of web development.
The reader should understand the meaning of HTTP~\cite{rfc7231}, HTML~\cite{hickson2012html}, 
DOM~\cite{2000Document}, \js~\cite{ecmascript2011ecmascript} and CSS~\cite{css21}.

\webscaleoutfig{}

\section{Scalable Web Server Architectures}
\label{sec:websys}

Web systems are system platforms that host web applications.
For example, a web system could be a machine running a web
server software serving static contents.

Scalability is an important issue for web systems.
Even in the early 90s, the first web system experienced a load rose by a factor
of 10 every year~\cite{berners1998world}.
Even without the increasing user access,
web applications could demand more resource as they become more sophisticated.
Figure~\ref{fig:webscaleout} is a diagram of scalable web systems.
In a scalable web system, web applications are able to efficiently 
harness resource from a cluster of servers.
We divide the servers in such web systems into two layers.
The web layer processes the HTTP requests from users
and generate HTTP response which will be rendered in users' browser.
The web layer offload storage and computation tasks to a storage layer
consists of database servers and other types of servers.
The benefit of layered design is manifold:
it makes the web servers light weight so they could handle more concurrent 
connections;
the whole system becomes modular,
the system owner could upgrade parts of the system without interfering
other parts.
Besides the web layer and the storage layer,
the load balancer is another important part of a web system.
The load balancer sits between the clients and the web layer,
it  distributes user's requests web servers
and return the response from the web servers to users.
From the user's point of view, 
he only needs one identifier to
access the web system, that could be a DNS~\cite{rfc1034} name or an IP address.
It is the load balancer's job to hide 
the distributed architecture
 from the user.
There are many ways to implement load balancer~\cite{cardellini2002state}.
For instance, 
we could
deploy reverse proxy software (like nginx~\cite{nginx})
on a group of servers
and configure the system's DNS name pointing to those servers,
then all the client traffic will be directed to reverse proxy servers
and reverse proxy will distribute the traffic to the web layer.
It is also possible that the load balancer layer is implemented purely in logic 
with no dedicated servers.
One way to do it is
 registering all the web servers' public IP addresses to the DNS server
and rely on the DNS server return these IP addresses to clients in a round-robin
fashion.
From now on, we will only discuss the distribution algorithm without meddling too much
in the implementation details of the load balancer layer.

% \subsection{Application State Management}% maybe too big a title for a short review...

% We use application state to describe 
% the data web applications need to 
% process user requests and render user interface.
% Based on the design of the web application,
% the application state could comprise data fetched from some permanent
% storage or 
% generated during the execution of the application.
% On the other direction,
% the data in application state could be written back to permanent
% storage or just be kept transiently.
% The application state data that stored permanently could
% endure server restart and be referenced in the future,
% it usually contains data that is costly or impossible to 
% recreate, like user profile, shopping transactions, etc.
% The application state data that is transient 
% usually represents state that is acceptable to be lost
% during server restart or client crash.


% To distinguish individual users, the web application assigns 
% a unique session identifier for each user.
% The session identifier is then attached to every HTTP requests.


\subsection{Session State Management}

We use session state to describe the application context
web applications need to maintain for each user.
It is generally acceptable to lose session state
when server restarts,
because session state could be recreated from scratch.
Session state is partitioned on client side and server side.
Usually the web application would assign a unique 
session id for each user and uses
the session id to associate the client side session state
with server side.
Most web frameworks provides two modes
to manages session state~\cite{j2eedoc}~\cite{phpdoc}:
a local mode where server side session data is stored
in memory or in file system,
a distributed mode where server side session data
is stored in a centralized storage system usually a
relational database.
It is easy to see that the local mode has better performance
than the distributed mode.


\subsection{Load Balance}

In this section we discuss the design choices of load balancer in a web system
and how the implementation of load balancer impacts the application development.

We use the taxonomy from the survey of Cardellini et al.~\cite{cardellini2002state}
to categorize the load balancer into client-blind or client-aware based on if
the load balancer uses any information from client request to perform the dispatch.

A client-blind load balancer does not use information from client.
For example, it could randomly select a web server from the web layer for each request,
or use a round robin algorithm to pick web server one by one.
In this case, the requests from the same client could be relayed to any web server.
The developer cannot naively use a local mode session implementation
, because as the requests from a user hit multiple servers,
multiple copies of session state would be created on these servers,
the discrepancy in these copies could cause application bugs not to mention
it is a waste of space to store multiple copies.
The application has to move session state to a centralized
storage system and fetch the session state for every request.
One solution would be move as much session state to the client side as possible,
then the client would attach its part of session state
with the requests to reduce the cost on the server side.
However, the server side still needs to keep part of the session state
because it is risky to store all session state on
the client side.
For example,
the server might need the client's user profile to check if the client has
the privilege to do a certain task.
It is obvious that server should not trust the user profile passed by the client
as a misbehaved client could forge a user profile in its requests.
Another solution would be synchronizing local mode session data among web servers
as implemented in Tomcat's session replication mode~\cite{tomcatcluster}. % and JBOSS
This solution is expansive because it needs to broadcast
changed session data to all web servers.

A client-aware load balancer dispatches the request from the same
client to the same web server.
In this model, the web server could use the local mode session implementation.
The downside is the load balancer needs to discriminate each client
to make the correct routing decision.
One way to do it is parsing the HTTP request header to get
the session identifier,
then the load balancer computes a hash value of
the session identifier and map the hash value to a web server.

% Another aspect that needs to be consider to design a scalable web system
% is the servers' load.
% The servers' load could be drastically uneven if the load balancer
% does not take the servers' state into the distribution process.

% Current web frameworks usually only deals with the Client Layer and the Web Layer,
% the web application developer should take the distribution policy of the load balancer
% into consideration when designing web applications,
% and they would need to configure or implement the load balancer to support
% the needs of the application.


% Current web frameworks do not have clear construct to define application state.
% One option for the developer is to rely on session objects provided by frameworks.
% The application developer needs to configure the persistence  of session objects
% and make sure the configuration works with the load distribution scheme.
% For example, in J2EE the session objects could be configured to be managed in memory or
% high available database~\cite{j2eedoc};
% in PHP developers could specify a session save handler to store session objects
% in database~\cite{phpdoc}.
% By default, these frameworks store session objects in memory or in local file system.
% If the developer adopts a client-blind load balancer,
% he needs to configure the session objects to use database or other
% synchronization mechanism to make sure the application state could be fetched on every server.
% Another option is to implement his own way of handling application state.
% Regardless of the choices of managing application state,
% the developer needs take the distribution policy of the load balancer
% into consideration when designing web applications,
% and in the other direction,
% the developer needs to find the right load balancer to support his application.






\section{\nodejs}

\cb is written in \nodejs.
\nodejs~\cite{tilkov2010node} is a platform that built on Google Chrome's V8~\cite{v8} \js engine.
Simply put, \nodejs is a standalone \js runtime that allows \js code to be
executed without a web browser.
It is also regarded as server side \js in contrast to typical \js code that executed in
client side web browsers.
\nodejs comes with a standard library that provides API
such as file system access, network IO, binary data manipulation, etc.
that makes it possible to write server side network applications using \js.
\nodejs is appealing for building high performance scalable web applications
because it adopts a non-blocking event driven IO model which makes it capable 
to handle a large number of concurrent connections.
\nodejs is widely adopted in the web development community,
a lot of third party packages are developed.
Another bonus for using \nodejs is that the \cb framework code and application code
are in the same language, there is language crossing cost.
In this section we will discuss the \nodejs's non-blocking IO model and some important
third party packages we used.

\subsection{Non-blocking IO}

\begin{listing}[ht,width=\columnwidth]
\begin{minted}[
frame=lines,
fontsize=\scriptsize,
linenos
]
{javascript}
var fs = require('fs');
fs.readFile('/etc/passwd', function (err, data) {
  if (err) throw err;
  console.log(data);
});
console.log("Reading file");
\end{minted}
\caption{Reading file and printing the content on console using \nodejs.}
\label{code:nodefile}
\end{listing}

Listing~\ref{code:nodefile} shows the non-block IO concept in \nodejs.
Because the IO API such as \emph{readFile} does not
block, line 6 will be executed before line 3.
To execute code after the IO operation completes,
the developer needs to register a callback function to be fired asynchronously.
In this example, after the file's content is read the callback function is placed on the
event loop,
then the runtime would dequeue this callback sometime in the future and execute it.
Like in a web browser, \nodejs execute \js code in a single thread,
and the runtime will not halt the current execution in the middle to execute something in
the event loop, so it is easy to write concurrent safe programs.
The event based non-blocking IO and simple execution semantics
is key to the performance of \nodejs,
the downside is that any operation that waits for IO must be implemented as a callback.

\subsection{Node-http-proxy}

Node-http-proxy~\cite{nodeproxy} is a \nodejs package for developing HTTP proxies.
It provides programming interface to relay a HTTP request to another server and
automatically copy the response from that server to the original requester.
It also supports WebSocket~\cite{rfc6455} protocol.
WebSocket is a TCP-based protocol which provides bidirectional full-duplex communication
mechanism for web browsers.
Using WebSocket, the server could send content to the client without requested
by the client.
Because WebSocket sends messages while keep the connection open,
it is cost efficient comparing to using HTTP requests.
We use WebSocket to implement DOM synchronization between client and server so
the server could push DOM changes to clients.
In node-http-proxy,
the programmer needs only to provide the destination server on the hand shake
request and all the subsequent WebSocket messages will
be transparently proxyed to the destination server.

\subsection{Jsdom}

Jsdom~\cite{JSDOM} is a \nodejs package which provides a \js implementation of 
DOM API. We use jsdom to implement server side DOM tree.
Jsdom works like a web browser except it does not render the DOM tree:
it reads HTML documents to build the initial DOM tree,
it also executes \js files specified in script tags.
It is possible to create multiple jsdom instances in a same process,
each of these jsdom instance will have its own DOM tree and
script execution environment,
the scripts in each jsdom instance will have its own isolated view of global variables.
In our model, each virtual browser has one jsdom instance and
forwards client events to the jsdom DOM tree.
We also patched jsdom to get notifications for DOM updates.


\section{\cb}

\architectureoverview{}

In this section we sketch the implementation of
the original single process version \cb{}~\cite{mcdaniel2012cloudbrowser}.
Figure \ref{fig:cb1arch} shows the relationship
between the client engine running in the user's browser and the virtual browser
running server side.  When the user visits the application, the client engine
code is downloaded and restores the current view of the application by
copying the current state of the server document.  Subsequently, user input
is captured, forwarded to the server engine inside the virtual browser,
which then dispatches events to the document.  All application logic runs
in the global scope associated with the virtual browser's window object.
Since the server environment faithfully mimics a real browser, libraries
such as AngularJS~\cite{hevery2009angular} can be used unchanged to implement the user interface.
Client and server communicate through a lightweight RPC protocol that is
layered on top of a bidirectional WebSocket communication.
Stylesheets, images, etc. are provided to the client through a resource
proxy.

\subsection{Deployment Model}
\label{sec:deploymodel}
\appbundlefig{}
As shown in Figure~\ref{fig:appbundle}, 
a \cb application bundle is a directory
contains descriptor files, an entry point HTML file,
\js files and resources files like CSS files, images, etc.
The descriptor files specify the application's name, owner, mount point, and
other application's configuration.
Mount point is the URL path to the application, for example,
if a \cb is deployed at \url{example.com} and an application's mount point is
\emph{chat}, the user could access the application at \url{http://example.com/chat}.
The \js files include libraries like AngularJS or JQuery
, the application code and an optional application instance(explained 
in Section~\ref{sec:appins}) definition.
Like in regular browser, the entry point HTML should have script tags
specifying the \js files the developer wants to be executed in virtual browsers.

The developer could put the application bundle in \cb's application 
directory or upload the bundle using \cb's administrator application.
Multiple applications could be deployed simultaneously.

\apphierarchyfig{}

As emphasized in Figure~\ref{fig:appidhierarchy},
an application could create multiple application instances,
each application instance could create multiple virtual browsers,
a virtual browser could be simultaneously accessed by multiple clients.
% For example, in our chat room example application, multiple clients
% can join one chat room and each client get his own view of the
% chat room.
% In our model, the chat room concept is represented by application instance,
% the user's view of chat rooms are represented by virtual browsers.
% When a user request the chat room application's URL,
% he is redirected to a page that lists all the chat rooms(in essential application instances) 
% he joined.
% From there, the user could manage his application instances.
% He could create a new application instance(chat room) and share the application
% instance's URL to let others join in.
% When a user joined a chat room,
% a virtual browser is created for him so he could interact with the application.
% A virtual browser could also be accessed by multiple clients,
% the DOM updates from the virtual browser will be broadcast to all connecting
% clients to provide a co-browsing experience.



\subsection{Application Instance}
\label{sec:appins}
\appinstancefig{}


Application instance allows multiple virtual browsers to share data structure.
As in Figure~\ref{fig:appinstance} shows, 
every virtual browser is created inside an application instance
and virtual browsers inside an application instance share application instance object.
The application instance itself
consists metadata about the application instance object
 such as ownership and access permissions.
The application instance object is defined
the application instance definition file 
in the application bundle (see Figure~\ref{fig:appbundle}).
When a virtual browser is created in the application instance,
the application instance object is injected into that virtual browser
and the application code could reference it directly.

\chatappfig{}

As an example, consider a scenario for a Chat application developed using AngularJS,
depicted in Figure~\ref{fig:chatapp}.
A system administrator of a \cb deployment would install the application, which give users the
ability to create application instances. To start a chat site, a user would create
an application instance and share its URL with chat participants.  As the participants join
the chat site, a virtual browser is created on demand for each participant, which is connected
to the application instance (the users can bookmark their virtual browser's URL to later return.)
The shared application instance data in such an application
consist of the chatroom(s), users and their associated messages.  The advantage of this design
is that AngularJS's dirty-checking mechanism will reflect updates to the shared instance data
in each virtual browsers' document automatically, thus ensuring that new message are broadcast
to each.



\subsection{Application Instantiation Strategy}
\label{sec:appinstantiation}

The application instantiation strategy specifies
how the application instance and virtual browsers are instantiated.

Application programmers can create CloudBrowser applications in the same way in which
they create the client-side portion of a client-centric application, using low or high level
JavaScript libraries such as jQuery~\cite{jquery} or AngularJS.  
A descriptor in the application's manifest describes 
their application's instantiation strategy.
The supported strategies include

\begin{description}

\item[singleAppInstance] The application supports only one instance and single virtual browser.
    All connected clients will share a single server-side document in this singleton - this can be
    used for applications that display data, such as a weather application. These applications will not
    typically react to user input and users do not need to be authenticated.

\item[singleUserInstance]  This application requires authentication to establish a
    user identity, which we provide through a local database as well as through external OpenID
    authentication.   In this mode, users may not create more than one virtual browser per
    application instance.  When a user accesses the application instance's URL, they will either
    be forwarded to their virtual browser or a virtual browser will be instantiated for them.

\item[multiInstance]
    Allows users to have multiple, separate virtual browsers connected to an application
    instance. For instance, a user may have to be in two separate chatrooms offered by one chat site.
    In those cases, the user has the largest flexibility, but will need to manage whether
    to join an existing virtual browser or create a new one when they visit the application instance
    - similar to the choice a user may have when deciding whether to navigate to a new site in
    an existing browser tab or open a new one.

\end{description}

Except for multiInstance applications, the existence of virtual browsers is not exposed to
end users that merely join existing application instances.


% The hierarchy that results from applications, application instances, and virtual browsers is
% depicted in Figure~\ref{fig:appidhierarchy}.  This figure shows the general case in which an
% application might allow multiple instances, and in which each user can create multiple virtual
% browsers.



\chapter{nodermi: A Remote Procedure Call Framework for \nodejs{}}
\label{ch:rmi}
\markright{nodermi}
In the single process version of \cb,
everything lives in the same address space,
which allows application code to invoke system components' functionality.
As \cbtwo divides \cb into multiple processes,
some of those local method invocations need to be replaced by
inter-process communication,
because as the framework spreads its state
and responsibility to multiple processes,
an operation initiated in one process
could require access to state located in another process.
For example,
in the single process version,
closing a virtual browser
is implemented by invoking the \code{close} method of the virtual browser
object.
If the virtual browser is in another process, closing it requires that a request be sent to that process.
We could have introduced new objects and methods to send specific messages,
but this approach is undesirable for the following reasons.
First, there are many call sites that would need to be changed.
Second, we would also have to design message formats for many different methods and
write handler code to parse and process each of these messages.
Third, since our codebase is continuously evolving, any changes to existing methods
would require modifications to the message communication layer.

To avoid having to manually create and maintain messaging code for each
method, we developed nodermi~\cite{nodermi}, an object-oriented
remote procedure call (RPC) framework for \nodejs{}.
% Nodermi hides the complexity of inter-process communication details from
% upper layer code and is adaptive to code changes.

Before we discuss the implementation of nodermi,
we first introduce some terms for nodermi and
RPC systems in general.
We use the term \emph{local object}
to refer to objects that are local to a calling process.
We use the term \emph{remote object} to represent objects that are
located in processes other than the calling process.
We use \emph{stub} or \emph{remote reference} to refer to special objects created
by nodermi that represents \emph{remote object}s in the calling process
(we can think of \emph{stub}s as proxies to remote objects).
We say an object is \emph{remotely referenced} from a process
if there are \emph{stub}s representing it.
Each \emph{stub} represents one \emph{remote object} and
it has methods that mirror the \emph{remote object}'s methods.
From the programmer's perspective,
\emph{stub}s appear like ordinary local objects.
When calling a \emph{stub}'s method, the underlying
RPC framework code sends messages to the \emph{remote object}'s process
to invoke the corresponding method there.
For a given stub,
the object it represents is called its \emph{source object},
the process that creates the \emph{source object}
is referred to as the \emph{host} of the stub. 
The \emph{host} process takes on the role of a \emph{server}, 
whereas the process that holds a \emph{stub} to the object act as a \emph{client}.
In \js{}, since functions are first class citizens, it is possible
to pass a function as an argument or return a function as a result.
Nodermi treats functions and objects alike so that functions can also be
referred to remotely.

\section{Semantics}
\label{sec:semantics}

\nodermiexamplefig{}

\paragraph{Synchronous Methods}
Unlike RPC frameworks in which a calling process blocks during the
completion of the remote method~\cite{birrell1984implementing},
nodermi is fully asynchronous since \js{} functions must not block.
In nodermi a remote method call returns immediately after initiating the communication,
similar to other IO-related asynchronous methods.
Thus the method's return value is not available when the stub method returns on the client.
To obtain the result of a remote method invocation,
the remote method must be implemented asynchronously,
passing its result(s) to the caller via a callback function.
If a caller requires the knowledge that a method has completed - with or without
returning a result, the method must be written in an asynchronous style.
Thus, some existing methods will need to be changed to be able to invoke them 
remotely via nodermi.

\nodermipassbyreffig{}


\nodrmipassbyvalfig{}


\paragraph{Passing Arguments}
Nodermi copies most arguments by value, including built-in primitive types
and objects.  If an object refers to other objects, these objects are copied 
as well, so that an object's entire transitive closure is copied.
Since JavaScript functions contain closure, it is not possible to copy them.
Instead, we create stubs for functions and methods before passing them
to the remote object's method.  If a method needs to invoke 
methods on objects passed as arguments, then such invocations
will cause the created stubs to initiate a remote method
call to the original object.  The client must create handlers
to receive and process these remote method invocations.
For instance, in Figure~\ref{fig:nodermiexample}, 
when \emph{Process B} calls the remote method \emph{method1}
in \emph{Process A} with a function argument,
a stub for the argument
\emph{callbackA} is created in \emph{Process A} because
it is not possible to copy function \emph{callbackA} to \emph{Process B}.
The stub to \emph{callbackA} is passed as an argument to
 \emph{method1}.
Then \emph{method1} invokes the stub to \emph{callbackA} as if it is a ordinary
 local function and the actual \emph{callbackA} is invoked in \emph{Process A}. 


For certain built-in types, e.g. \code{Date}, \code{Error}, \code{Buffer}, 
we copy their contents and recreate them via their constructors.
So the a method could call methods of these objects without 
remote method invocation.
In this case, no stubs is created as emphasized in Figure~\ref{fig:nodermipassbyval}.
Besides built-in types, nodermi also allows programmers to
register customized types to be copied and recreated via constructors.
The objects of these customized types need to implement a method to dump
their contents as constructor arguments.
This semantic is unlike Java RMI\cite{j2eedoc}, which loads
object's methods through a separate mechanism via Class Loaders,
we assume constructors of these objects are registered in
both sender process and receiver process.

Nodermi allows programmers pass stubs as arguments.
If we treat a stub like other normal objects and create a remote reference 
 for the stub when it is passed to a remote method. 
Then when calling a method of the newly created remote reference,
a remote method invocation will be initiated to call the corresponding method
of the stub, which will in turn initiate another remote method invocation
to call the object the stub represents.
The chain of remote method invocation initiated by a method call could
be arbitrary long.
To avoid this problem, when we pass a stub to a remote method call,
we create a new remote reference referencing 
 the stub's \emph{source object} instead of referencing the stub itself
 (See Figure~\ref{fig:nodermipassbyref}).
Then calling method on the new remote reference will initiate 
communication directly to the stub's host process.
An exception is that when the stub's host process is the server
process of the original remote method that we are passing arguments to,
it would be unnecessary to create a remote reference 
in the server process to reference one of its local object,
we use the stub's source object directly instead of 
creating a new remote reference.


\paragraph{Stub Property}
A stub's properties is a snapshot of the properties
of its source object.
We could read or write properties of a stub directly.
But the changes we made to a stub's property
will not be synchronized to its source object.
Vice versa,
changing the properties on the source object has no effect
on the stub either.
It is ideal to make the stub acts like a regular reference
to its source object such that
changes made on one side would be reflected on the other.
It is not feasible to implement this semantics because
in \js{} reading or assigning a object property is synchronous,
we cannot block to wait for synchronization takes effect
 during these operations.
We could have intercept assignment to existing properties using
\emph{setter} and propagate the assignment to other processes.
We still cannot preserve the semantics of the local property assignment
where the property update takes effect right after the assignment statement returns.
Moreover, there is no way in \js{} to intercept the operation of
adding a new property or deleting a property.

% TODO comparison...

\section{Design}

\nodermifig{}

As shown in Figure~\ref{fig:nodermi},
calling a stub method results in calling
\emph{remote method invocation} module in nodermi.
This module
first encodes the arguments of the method call
 and other necessary information
as a \emph{method invocation message}.
The transport layer then converts the message to
a binary string and sends it
to the server process of the remote method.
On the server side, the transport layer reads
the message from the network and invokes \emph{method invocation handler}.
The handler reads the description of the arguments 
 by decoding the \emph{method invocation message}.
Then it recreates the arguments according to 
the semantics of argument passing (see section~\ref{sec:semantics}).
Finally, the handler invokes the corresponding local method with the arguments
it constructs earlier.

To locate the host process and source object of a stub,
nodermi stores the host process's identifier and the source object's object id
inside the stub so these information could be retrieved later by 
the \emph{remote method invocation} module.
The process identifier is a host and port pair that each process
allocates for nodermi to listen incoming TCP message.
The object id is a unique id nodermi assigns for each object that
is remotely referenced.


Let's explain the message flow with the example
in Figure~\ref{fig:nodermiexample}.
As shown in the code snippet on the top left of the figure,
process A invokes \emph{method1} with a function argument \emph{callbackA}.
First nodermi creates a \emph{method invocation message}
contains the object id of \emph{objB} and the method being called.
The message also contains a description of the arguments:
the object id of \emph{callbackA} and
the type of \emph{callbackA}.
After process B receives the message,
it reconstructs the arguments for the method call:
creating stub \emph{stubCallbackA} for \emph{callbackA} since it is a function.
The stub \emph{stubCallbackA} is a function and
it stores \emph{callbackA}'s id and Process A's process identifier.
After process B finds \emph{objB} via object id,
process B invokes \emph{method1} with \emph{stubCallbackA} as argument.
Then the \emph{method1} then invokes \emph{stubCallbackA} with \emph{result}
which in turn initiate a remote method invocation to call \emph{callbackA} in process A.


\paragraph{Bootstrap}
It is clear that nodermi automatically creates stubs during
passing arguments in remote method invocation without programmers
explicitly calling framework API.
However, the first \emph{stub} cannot be obtained
via remote method invocation.
To bootstrap the RPC communication,
nodermi provides \code{registerObj} method
to register a local object with a name
and \code{retrieveObj} method to 
directly create a stub representing
a remote object that has been registered via
\code{registerObj}.

\paragraph{Message Encoding}
When passing an object from a process to another,
nodermi adopts several policies to minimize the size of messages:
Nodermi assumes properties whose names starts with ``_'' are private properties
and skip them during encoding;
Nodermi skips certain types of objects e.g. Sockets because
it makes no sense for these objects to be read remotely;
Nodermi also provides mechanism for programmers to specify which
properties could be ignored during encoding;
Finally, nodermi use Protobuf~\cite{protobuf} a very compact binary format 
to serialize its messages.


\section{Distributed Garbage Collection}
\js{} relies on garbage collection to reclaim memory taken by
objects that are not referenced in the program.
However, the garbage collector is not aware a local object
could be referenced by another process through nodermi.
In the example shown in Figure~\ref{fig:nodermiexample},
after \emph{method1} of \emph{stubB} is called,
\emph{callbackA} is out of scope in process A,
it is not referenced anymore by the user code.
If nodermi does not hold reference to \emph{callbackA},
it could be garbage collected before it is invoked.
In fact, in practice callback functions like \emph{callbackA} are almost
immediately garbage collected because it is likely that these functions
live on the new space of a generational garbage collector heap that is frequently
garbage collected.
However, if we naively keep reference to objects like
 \emph{callbackA} in nodermi,
our application is surely prone to memory leak
as objects that are once remotely referenced
will never be garbage collected even after the local and remote processes
no longer use them.
Setting a timeout to automatically clean references of
these objects from nodermi is not going to work either,
since there is no guarantee that when the client is going to use a
particular object.
For example,
when a process registers a listener function to a remote event publisher,
the listener could be remotely triggered at anytime in the future.


There are a lot of research related to garbage collection in a distributed
environment~\cite{abdullahi1998garbage}, ~\cite{birrell1993distributed}.
We designed a mechanism that is similar to the sequence reference counting algorithm
designed by Birrell et al.~\cite{birrell1993distributed}.
We will discuss the difference between our work and ~\cite{birrell1993distributed}
in section~\ref{sec:relatedrpc}.

The high level design of our distributed garbage collecting
algorithm works as follows.
For each process, nodermi has an object map containing the objects
that are still remotely referenced.
The object map prevents garbage collector from prematurely garbage collecting
these objects by holding references to them.
This map is also necessary for looking up local objects
when handling \emph{method invocation} messages.
On the other hand, nodermi holds weak references(implemented using node-weak~\cite{nodeweak})
to \emph{stub}s.
Garbage collector could collect objects that are only
referenced by weak references.
When a stub is not used, it would be garbage collected,
then a callback function which is registered when creating
the stub's weak reference is fired to
send the stub's host process a \emph{dereference message}.
Upon receiving a dereference message,
nodermi knows that a remote reference to a local object
is now defunct.
In this way, an object could be removed from the object map after
all of the stubs representing it being garbage collected.
At this point, if it is not referenced locally it could
be safely garbage collected.


\nodermiobjmapfig{}

Figure~\ref{fig:nodermiobjmap} shows the structure of
the object map and the stub map
where nodermi stores weak references to stubs.
The object map contains the local objects and information of
their remote references.
A remote reference is represented by a session id,
a unique id generated by nodermi every time it transmits objects
to remote processes.
When a local object is transmitted, the current session id is added to
the object map indicating a new remote reference is created.
On the receiving side, the session id is written to
the \emph{stub}s created in this transmission.
When a \emph{stub} is garbage collected,
the \emph{dereference message} will contain
the remote object's id and the session id stored
in the stub.


So far, we only considered the scenario of a process sending its local
objects to another process.
It is also possible that a process could send a \emph{stub} to another process.
If the \emph{stub}'s host is the receiver, then there is no problem because no remote
reference is created.
If the \emph{stub}'s host is not the receiver(it could not be the sender either),
then a record needs to be added to the host's object map to indicate
a new remote reference.
In this situation,
the sender process first sends a \emph{reference message} to
ask the host
to generate a new session id
and put that session id into the host's object map.
After that,
the sender will send the \emph{stub} and the new session id
to the receiver.
By the time the receiver gets the message from the sender
and
creates a new \emph{stub} to the original object,
the host already has the record of this new remote reference.

A process could terminate without explicitly releasing its remote references
by sending dereference messages.
This could create false remote references on other processes.
To remove these false remote references,
nodermi cleans a remote process's records in the object map if
it detects that remote process is terminated.
Nodermi keeps a table of last response time of remote processes it contacted,
it updates the table every time it receives a message from a remote process.
When the table entry for a process is not updated for more than 60 seconds, nodermi
will send a ping message to that process.
If the ping message is unanswered, all records of that process will be removed
from the object map.
This feature ensures that a terminated process will not cause memory leak on
other processes.




% FIXME comparison

% semantics of properties, methods, this  field accesses, object identity
% caja, druby, javascript , python

% \section{Issues}

% async
% network partition, transmission failure, server restart

\chapter{Implementation}
\label{ch:impl}
\markright{Implementation}

%Since virtual browsers occupy most of the system resources, the new
%distributed design spread the virtual browsers to multiple processes
%to improve the system's scalability.
\emph{put a paragraph in here saying the keypoints/highlights of what this chapter is about,
establishing a roadmap, keeps things in the order in which it'll be discussed.}

As shown in Figure~\ref{fig:cb2arch}, \cbtwo consists of a
single master process, multiple reverse proxies, and multiple worker
processes. All processes communicate via nodermi, which in turn uses standard TCP/IP 
sockets in its transport layer, so they can
be located on a shared-memory multiprocessor machine or on different machines in a cluster. 
Worker processes host application instances and virtual browsers. The master process is responsible for
the request dispatch logic which decides how to distribute the client load to
workers. The actual dispatching is implemented by the reverse proxies, which
forward users' requests to workers and copy workers' responses back to users.
All reverse proxy processes are bound to the socket that accepts client
requests, allowing the OS to distribute pending client connections in a round-
robin fashion.
The reverse proxy can relay both HTTP requests/responses as
well as the bidirectional WebSocket protocol (see Section~\ref{sec:nodepackage}).
Once the client has established a WebSocket connection with
the server side, the majority of traffic will be WebSocket
messages for which there is relatively little per-message overhead.

\newarchitectureoverview{}


% The master process is a single point of failure,
% but it is light weight and does less computation,
% so it is less likely to fail in practice.

% discuss the user request, message flow

\section{Request Dispatch}
\label{sec:reqdis}
% ---- motivation, use cases here
% ---- make it clear that after dispatch, the messages websocket

When a request is being dispatched, the reverse proxy uses information
contained in the request URL to make an appropriate forwarding decision. 
The system exposes three types of URLs for the users to access \cb applications.
The formats of these URLs are as follows:

\begin{description}

\item[Application URL] \label{itm:appurl} \hfill \\
Format: \url{http://example.com/[app]}, \code{[app]} represents an
application's mount point.   For example, if an application's mount point is
\emph{chat},  then its Application URL is \url{http://example.com/chat}.


\item[\appins{} URL] \label{itm:appinsurl} \hfill \\
Format: \url{http://example.com/[app]/a/[appInstanceId]},
\code{[appInstanceId]} represents an \appins{}'s id.  For example, if an
\appins{}'s id is \emph{appins1} and its application's mount point is
\emph{chat}, then its \appins{} URL is
\url{http://example.com/chat/a/appins1}.


\item[Browser URL] \label{itm:vburl} \hfill \\
Format: \url{http://example.com/[app]/a/[appInstanceId]/b/[browserId]},
\code{[browserId]} represents a virtual browser's id. For example, if a
virtual browser's id is \emph{browser1} and its \appins{} id is 
 \emph{appins1} and its application's mount point is \emph{chat},
  then the virtual browser's
Browser URL is \url{http://example.com/chat/a/appins1/b/browser1}.

\end{description}

% justify why you included /a and /b (for clarity?)

% -----scenario of these URLs, when low load, the reverse proxy could be embedded

For all application instantiation strategies (discussed in
Section~\ref{sec:appinstantiation}), the user may prefer to access
the application using the \emph{Application URL} because it is easy to
remember and it leads to a default view where user can navigate to other
views (if he has any).  
From the user's point of view,
the \emph{Application URL} is similar to the homepage URL in a traditional web application.

\emph{the above sounds vague}

For \emph{singleAppInstance}   and \emph{singleInstancePerUser}
applications where the user can access only one virtual browser, the system
redirects the user to that virtual browser when handling \emph{Application URL}. 
For \emph{singleBrowserPerUser} and \emph{multiInstance} applications, the
system will present the user a landing page listing all his \appins{}s and
virtual browsers when handling \emph{Application URL}. From there the user can
access any of his \appins{}.

\emph{include screenshot of landing page}.

\emph{\appins{} URLs} are used for joining a specific, already created \appins{}.  
For \emph{singleBrowserPerUser} applications where each user
can create only  one virtual browser for a given \appins{}, the system directs
the user to his own virtual browser in the specified \appins{} when handling
\emph{\appins{} URL}. 

For \emph{multiInstance} applications, it is the user's responsibility
to decide whether they wish to join an existing virtual browser or create a new one,
the system will lead the user to a landing page to make that decision.
For \emph{singleBrowserPerUser} and \emph{multiInstance} applications,
users can share an \appins{} with others by sharing its \emph{\appins{} URL}.

\appins{} URLs are not meaningful for \emph{singleAppInstance}   and
\emph{singleInstancePerUser} applications because there is only one \appins{}
for a given user. 

\emph{Browser URLs} are for connecting to a specific virtual browser. A user can
bookmark a \emph{Browser URL} so he can revisit that specific  virtual browser
in the future. The user can also share a virtual browser with other people by
sharing its \emph{Browser URL}.

As discussed in Section~\ref{sec:deploymodel}, multiple virtual browsers
may share data structures belonging to an \appins{}.
\cbtwo colocates every \appins and its virtual browsers in the same worker process.
Once an \appins has been created in a particular worker process, all future
requests to \emph{\appins{} URL}s and \emph{Browser URL}s for that instance's
ID must be routed to that worker.

\emph{ say something like when a app/inst or vb is created, user is redirected
... maybe this is already described below?}

%Application URL}s, the system need first allocate an \appins{} and virtual
%browser to handle the request according to the instantiation strategy, and then
%redirect the request with the virtual browser's URL.

Although the reverse proxy processes are in charge of the actual forwarding,
the master process keeps track of the forwarding map. Thus,
when a client sends a HTTP request, a reverse proxy process
will accept this HTTP request and ask the master where to dispatch this request. 

If the request URL is a \emph{Browser URL} or \emph{\appins URL}, 
the master extracts \appins id from the URL and returns the associated
worker via a \appins id to worker lookup table.
This lookup table is updated every time an \appins{} is created or removed.
For an \emph{\appins URL} request, the worker will continue to find the associated
virtual browser for the user and send back a redirect with the virtual browser's URL.

If the request URL is an \emph{Application URL}, 
the system's behavior varies according to the application's instantiation strategy.
For example, for a \emph{singleInstancePerUser} application,
the system can either create a new \appins{} for the user if the user does not have an
\appins or use the user's existing \appins{} if otherwise.
The instantiation strategy specific logic is handled by workers, not the master,
because it requires access to authentication information.
Thus, the master can pick any worker using a load
balance algorithm (detailed in Section~\ref{sec:lb}),
and the selected worker will then process the request according
to the instantiation strategy.
After consulting the authentication information, this worker will redirect 
the client to a specific Browser URL.   The virtual browser corresponding
to that URL may be located in another worker process.

When a worker process receives a Browser URL request, it will locate the
corresponding virtual browser and responds with an HTML document that contains
information to bootstrap the client engine. 
The client engine will create a WebSocket connection to establish an RPC channel to
the worker process.
First, the client engine sends a WebSocket handshake request to initiate the
WebSocket connection.
The handshake request URL contains the associated \appins id and virtual browser
id so the master can find the corresponding worker for this request.
After the worker sends back the handshake response, the connection
between the worker and the reverse proxy that performed the handshake, 
as well as the connection between the reverse proxy and the user remains open. 
The reverse proxy will relay all subsequent WebSocket messages
without parsing them and without requiring repeated calls to the master
to obtain forwarding information.

As the client engine renders the client-side view of the page, it may trigger
requests for auxiliary resources such as images or spreadsheets. 
These requests will contain the \appins{} id in the URL, which the master
uses to forward those requests to the appropriate worker.

%If any exception happens, for example, the user sends a Browser URL with no
%corresponding virtual browser in the system, the system will reject the
%request with an error message.

In this design, the reverse proxies ask the master for the dispatch decision
for every HTTP request.  However, HTTP requests are required only when users
reconnect to virtual browsers - most of the actual interaction with a virtual
browser is performed using RPC messages carried over WebSocket connections.
Thus, the overhead of inter-process communication between the reverse proxies and
the master to obtain forwarding information affects only a small portion of the network traffic. 

We also provide an option to embed a reverse proxy instance inside the master process.
If the system is configured with one single embedded reverse proxy,
reverse proxy and master can communicate directly.
However, in this mode, the system can support fewer concurrent users than 
when using multiple reverse proxies.

\section{Load Balancing}
\label{sec:lb}

The load balancing algorithm is invoked in two scenarios:
First, when the user requests an \emph{Application URL},
the master needs to find a worker to handle this request.
Second, it is invoked when the system is about to create a \appins{}.
Although such action can originate in any worker,
the load balancing algorithm is performed by the master.

For other user requests the routing is determined by the \appins{}-to-worker map
so there is no need for load balancing.
In particular, the creation of virtual browsers is not subject to load balancing.
Virtual browsers have to be placed in the same worker that hosts its \appins{}.

We support two load balancing strategies: first, the master can assign the
load to workers in a simple round-robin fashion. However, since \appins{}s may
vary widely in terms of the actual cost they impose on a worker and \appins{}s
can be terminated, the round-robin assignment works well only for cases in 
which resource use is uniformly distributed.
We also implemented a load-based scheme in which
workers periodically report a measure of current load to the master. The
master will select the worker with the lowest load when making load balancing decisions.
We have found the amount of heap memory that is currently in use a good measure
of a worker's momentary load.

In the load-based mode, the master's knowledge of a worker's load is not
always up-to-date as the worker's load can change before the master receives the
next report from the worker. We have found that this can lead to very unbalanced load
distributions. For example, when the system is creating a burst of \appins{}s,
they are all assigned to the same worker that has the lowest load at that
moment and this worker ends up hosting a disproportionate load.  To mitigate
this issue,  after the load balancing algorithm selects a worker, the
algorithm makes a projection of the worker's load after accepting the new
load. The projected value is used as the worker's load value  until the master
sees the worker's next report.  We have found it unnecessary to exactly 
predict the amount of incremental load as long as we do not significantly 
underpredict.  Based on our evaluation, a virtual
browser in a non-trivial application takes about 6M.  The master projects a
worker's load increase by 10M for every new \appins assignment assuming every
\appins will create 1 or 2 virtual browsers.  For most cases, the master
overestimates the actual load increase. It is not a problem because the master
corrects its estimate as soon as it sees the worker's next report. 
% ???
%In our experience, even under bursts of new requests the load distribution among the
%workers does not vary greatly.

\section{Master Implementation}

When the master process starts up, it initiates the following modules:

\begin{description}
\item[Application Manager] \hfill \\
Application Manager is responsible to maintain the state of applications. It
reads the application bundles (discussed in Section~\ref{sec:deploymodel}) and
initiate data structures to represent applications.

\item[Worker Manager] \hfill \\
Worker manager is responsible for the request dispatch
and the load balance algorithm (detailed in Section~\ref{sec:lb}).

\item[Reverse Proxy Manager] \hfill \\
Reverse proxy manager starts multiple reverse proxy processes as child
processes of the master and handles queries from them.
% the queries are sent from pipe, called ipc channel in node.js
% https://github.com/joyent/node/blob/master/lib/child_process.js
\end{description}

After initiation, the master exposes \emph{Application Manager} and
\emph{Worker Manager} to the workers via nodermi's \code{registerObj}  method
(discussed in Chapter~\ref{ch:rmi}).

The \emph{Application Manager} has application management related  methods,
such as list applications, create a new application, remove an application,
etc.  The applications are represented by application objects. An application
object has methods to maintain the internal state of the application, such as
change the application's authentication policy, register a new \appins, remove
an \appins, etc.  Since \emph{Application Manager} is registered in nodermi,
the worker can remote reference to it and get remote references to application
objects by calling its methods remotely. Via remote method invocation of
\emph{Application Manager} and application objects, the worker can get
metadata of all applications and register \appins{}s.  The worker also passes
reference to its own \emph{Application Manager}  and application objects to
the master's \emph{Application Manager} via remote method invocation.  The
master \emph{Application Manager}  uses remote references to these worker
objects to push application-related changes to the  workers.  For instance,
the master calls  the worker's  \code{addApplication} method remotely   to
push new application to that worker and remove an application from that worker
by calling the worker's \code{removeApplication} method remotely.

The \emph{Worker Manager} contains methods for worker to register its HTTP
hostname/port and report its load. It also maintains a table of \appins{} id
to worker HTTP hostname/port. This table is updated every time a \appins{}
is registered or removed from the \emph{Application Manager}.




\section{Worker Implementation}
\label{sec:worker}

For each worker, the system administrator needs to specify a hostname/port
pair the worker listens for HTTP requests,  the master's nodermi instance's
hostname/port, and a hostname/port for its own nodermi instance in a
configuration file. To ease the deployment effort, we provide a tool that can
generate multiple worker configuration files. Based on the information in the
configuration file, the worker creates a HTTP server and a nodermi instance.
After creating its internal modules, the worker process obtains remote
references to the master's \emph{Application Manager} and \emph{Worker
Manager} via nodermi, then  it fetches all the applications' metadata and
registers itself to the master by invoking methods of these remote references.
If the master is not running, the worker will keep on retrying until these
remote references are obtained.


The worker process maintains the mappings of \appins{} ids to its local
\appins{}s and browser ids to its local virtual browsers.
It tries to handle incoming HTTP requests with its local virtual browser
if possible.
When a worker receives a HTTP request, it parses the request to extract
\appins id, virtual browser id and get associated user information
from session. If the request does not
have an \appins id, the worker will find an existing virtual browser in the
system or tell the master to create a new one based on the application
instantiation strategy. After
that, the worker sends back a redirect response with the virtual browser's
\emph{Browser URL}. If the request has an \appins id, the worker first check
if the id matches any local \appins and rejects the request if the check
fails. If the request does not have a virtual browser id, the worker either
picks an existing virtual browser in the \appins  or create a new one based on
the application instantiation strategy, then the worker sends back a redirect
response with  the virtual browser's \emph{Browser URL}. If the request has a
virtual browser id and it matches a virtual browser in the \appins, the worker
processes the request based on the type of the request: For \emph{Browser URL}
requests, the worker sends back the initial  HTML document; For resource
requests, the virtual browser will read the corresponding local resource files
from file system or from Internet, and send back the content. For WebSocket
handshake requests, the worker would send back a handshake response message
and add the WebSocket connection to the virtual browser.
The subsequent WebSocket messages will be directly handled by the virtual browser.


% If the application requires authorization in the configuration, the
% worker would also check if the user has the permission to access the \appins
% or virtual browser specified in the request. The framework has a default
% permission policy that is useful for most cases. For example, by default the
% user could only access the virtual browsers he creates. We also provide a
% default web interface for users to manage the permissions of their \appins{}s
% and virtual browsers. The programmer could also explicitly manage user
% permissions in application code.

\section{Secure Access to Framework Object}
\label{sec:api}

There are many use cases when an application needs to access framework's
internal objects. For administrator applications that monitor and manage the
system, nearly every feature requires access to the framework's internals. For
instance, in our administration dashboard one can view all the installed
applications and upload new applications. For general purpose applications,
some common tasks could only be implemented by the framework. For example, in
a chat room application, a user can terminate a chat room (i.e. application
instance) and close all its associated chat windows (i.e. virtual browser).

We designed an API layer such that the applications can only call the
framework methods via the API. We do not allow the applications call framework
code directly for the following reasons:

\begin{description}

\item[Security]  We want to enforce permission check for users' actions so that
they do not violate the permission settings (discussed in
Section~\ref{sec:worker}). For example, only the system administrator is
allowed to shutdown all the  applications. 
If we allow the application use the internal API directly,
the application can allow any user do anything.


\item[Isolation]  First, we want to prevent the application code from
manipulating the internal objects in unexpected ways. Second, we want to
decouple the application code and the framework code so  that changes made to
the framework code do not break the applications.   Third,   we want the
framework objects to be available for garbage collection after they are no
longer needed by the framework,  and the objects created by application code
to be available for garbage collection after their associated virtual browsers
and \appins{}s are closed.  If the applications are allowed to access the
framework objects directly,   it is impossible to fulfill these requirements
since the objects created by application code and framework objects can create
references to each other.

\end{description}


\subsection{Overall Design}

The \cb API has four major classes covering essential features applications
need to access framework objects.

\begin{description}

\item[APIBrowser] Represents a virtual browser.
It has methods to manipulate the associated virtual browser, such as
set the browser's permissions, close the browser, etc.

\item[APIAppInstance] Represents an \appins{}. It has methods to manipulate
the associated \appins{} and list the \appins{}'s virtual browsers as 
\emph{APIBrowser} objects.


\item[APIApplication] Represents an application object.
It has methods to manipulate the associated application
and list the application's \appins{}s as \emph{APIAppInstance} objects.


\item[APICloudBrowser]
It has methods to manage applications and list applications as \emph{APIApplication} objects.
For every virtual browser one single APICloudBrowser object is created and injected as
a global variable in the application code namespace.

\end{description}

\apiclassfig{}


Initially, the application code can get a reference to a
\emph{APICloudBrowser} object. As shown in Figure~\ref{fig:apiclass}, from
this object, the application code can get API objects representing
applications, application instances and virtual browsers. 
For most cases, the
current application object, the current \appins{} and the current virtual browser are
the most frequently visited framework objects for the application code, thus
the \emph{APICloudBrowser} object also contains API object properties
represent these objects so that  the application code can get reference to
these API objects without calling a chain of methods. Besides framework
objects, the \emph{APICloudBrowser} object also contains properties that
provide auxiliary services such as sending emails.


\subsection{Implementation}

API objects are implemented in proxy design pattern :  when their methods are
called, they will first perform permission checking if necessary and then call
the corresponding methods of the framework objects they represent. Proxy
design pattern requires the proxy objects keep references to the objects they
represent. Typically these references are implemented by private properties so
that  the caller of the proxy objects cannot get reference to the the objects
being represented. Since there there is no language support for private
properties in \js, we cannot  store the internal objects as properties of the
API objects because the  application code can get reference to internal
objects via reading properties. Even we use  hidden non-enumerable property
the application code can still guess the property name.

We use closure visibility to implement an equivalence of private property as
shown in Listing~\ref{code:apiconstructor}. We store internal objects as local
variables in the API objects' constructors and define API methods inside the
constructors. Then in the API methods we can reference internal objects
directly because the API methods and the variables referencing the internal
objects are in the same closure. The application code cannot get the
references to internal objects because they cannot be read as properties and they
cannot be revealed by reflection.


\begin{listing}[ht,width=\columnwidth]
\begin{minted}[
frame=lines,
fontsize=\scriptsize,
linenos
]
{javascript}
// constructor. virtualBrowser is the actual virtual browser object
function APICloudBrowser(virtualBrowser){
    // store the internal object as a local variable
    var virtualbrowser = virtualBrowser;
    // define API methods here
    this.close = function(){
        // check if the user has the privilege to close this virtual browser
        privilegeCheckingCode();
        // forward the operation to the internal object
        virtualbrowser.close();
    };
}
\end{minted}
\caption{Snippet of Class APIBrowser: the API Class for virtual browsers}
\label{code:apiconstructor}
\end{listing}

To avoid holding references to the internal objects that are no longer needed
by the framework, the API objects only keep weak references to the internal
objects(See Figure~\ref{fig:apireference}). We use a \nodejs{} package node-
weak~\cite{nodeweak} to implement weak reference. Unlike ordinary references,
the objects referenced only by weak references are free to be garbage
collected.
In this way, when an internal object is no longer referenced
inside the framework, it is available for garbage collector even it is still
referenced by API.


\apireferencefig{}


The framework can also obtain references to the objects created by application
code.  Right now these references are created by the event register methods
provided by API for the application to register event listeners to internal
objects. For example, the application can register a event listener to notify
the current user when someone shares an virtual browser to him.  Because these
listeners are registered via API, the API layer is able to remember these
listeners. When a virtual browser's \code{close} method is called, the API
unregisters the event listeners created by the virtual browser. In this
way, when a virtual browser is closed, the framework won't keep reference to
the objects it created.


\section{Evaluation}
\label{sec:eval}
We have wrote several web applications on top of \cb{} to test its
performance and scalability.
One application is called click application,
it increments and prints a counter on HTML whenever the user clicks.
This application represents applications with simple user interactions and small amount of DOM elements.
Other applications are chat applications.
A user can join a chat room to chat with other people,
changes his nickname, send messages, view recent messages in the chat room.
Whenever the number of messages in a chat room reaches 100, the first 50 messages
are discarded.
The chat applications represents more sophisticated applications.
We implemented the chat applications use popular web frameworks Angular and JQuery
to assess how \cb{} performs with real world web technology stack.

We implemented a benchmark tool to simulate multiple users.
The benchmark tool will send HTTP requests to mimic the bootstrapping process,
and send web socket message to \cb{} to simulate user interaction with the application.
These messages are identical to those triggered by real user events like mouse click and key stroke.

In the benchmarks, each simulated client will wait for its action taking effect before 
sending the next one.
For example, in the click application, the client will wait for the counter
to be refreshed before clicking again.
In the chat application, the client will wait for
the chat message appearing on the chat window before sending another one.
% TBD

The \cb{} system is deployed on a server with 8 Intel 2.27GHz cores and 12G memory.
The benchmark tool is deployed on a separate machine.

\subsection{Click Application}
In this benchmark,
we allocate a separate virtual browser for each client,
a client will click on the page, then wait for the counter to be refreshed and click again.

\clickthroughput{}
\clicklatency{}

\clickwaitthroughput{}
\clickwaitlatency{}

\subsection{Angular Chat Application}
In this benchmark,
every 5 clients will share a chat room, 
after join the chat room, the client will change its nickname,
after that the client will start send messages.
The client will wait for each action takes effect to start the next one.
For example, the client will wait for his previous message rendered on the chat window before
sending the next one.


\subsection{JQuery Chat Application}

\jquerychatlatency{}
\chapter{Related Work}
\markright{Related Work}

\section{Remote Procedure Call(RPC) Frameworks}
Some remote procedure call frameworks deals similar problems as nodermi,
such as object oriented remote method calls,
distributed garbage collection, etc.

Dnode~\cite{dnode} is a RPC framework for \nodejs{}. 
Like nodermi, it is object oriented and it allows remote objects to be 
returned in a function or be passed as function parameters.
However, it does not distinguish remote stubs from local objects 
when marshaling them into messages.
As a consequence, a remote stub could point to a remote stub on another process,
a remote method call can cause a long chain of request round trips.
Even when a process is passing a remote stub is referencing a local object in the client,
the client will also create a stub point to the stub that is pointing to its local object.
Because of it does not deal with the problem brought by copying remote reference,
the garbage collection scheme is straightforward in dnode.
Dnode keeps reference to the object in a reference map when the object
 is serialized as a message,
on the client side,
whenever a remote reference is garbage collected, 
a \emph{cull} message(similar to our dereference message) 
is sent to inform the host to remove the object from 
the reference map.
There is a reference map for every client so the \emph{cull} messages from
different clients do not conflict with each other.
 
Pyro~\cite{pyro} is a object oriented RPC framework for python programming language.
Because python supports interception of creating, deleting and changing object properties,
and because python is not a ``non-blocking'' language like \js{},
pyro is able to support property manipulation on remote stubs in the exact same semantics as manipulating 
properties of local objects.
The downside of pyro is that every object needs to be registered in pyro manually
if you want it to be passed to another process as remote reference.
If a object is not registered,
pyro will serialize it as a \emph{dict} structure, i.e., a hash map equivalent data structure in python.
Also, there is no garbage collection mechanism in pyro, 
objects are kept in server by pyro even no client or other server component is referencing them.

Druby(drb)~\cite{druby} is a object oriented RPC framework that is built in ruby's standard library.
Druby has two modes of transmitting a object to client : by value and by reference.
If a object is transferred by value, the client will create a copy of that object and method
invocation on that object will only take place on the client.
The by reference mode creates a remote stub on the client side, and the stub will proxy 
method calls to the server.
Like pyro, it is the programmer's responsibility to explicitly define if a object could be
remotely referenced.
It also requires user to explicitly hold reference to remote objects to prevent them from being 
prematurely garbage collected. 

Network objects~\cite{birrell1993distributed} is a distribute programming paradigm
that provides RPC in a object oriented style.
The major difference between network objects and nodermi is that 
a client cannot directly read or write a remote object's properties,
and a client only keeps one stub per remote object.
As we explained in %FIXME

\chapter{Conclusion}
\markright{Conclusion}

\section{Future Work}

We plan to continue optimizing our system by incorporating future updates of
\js language, \js runtime engine and third party dependencies. The upcoming
\js standard brings a lot of promising features.  One of the new language
feature in ECMAScript 7 proposal is the native support for observing changes
to objects~\cite{jsobserveprop}. A evaluation shows that dirty checking in
AngularJS becomes 20 to 40 times faster using this
feature~\cite{angularjsspeedup}. This will greatly improve our system's
capacity to host AngularJS applications.


% improve dependency packages... 

We also plan to implement failover mechanism for   worker processes.  In
current implementation, when a worker process terminates,  the master still
dispatches requests to that worker. We need to implement a failure detection
mechanism to detect failed workers and a method to migrate the failed workers'
\appins{}s and virtual browsers to living workers.


Finally, as a PaaS platform we need a better admin interface 
to enable developers to monitor and diagnose their applications.
Currently, the admin interface only displays very basic information for deployed 
applications. We need to expose application logs 
and performance related statistics to application developers.

\section{Summary}

Maintaining the rich presentation state  of a web application server-side has
many potential benefits, ranging from simplified application development,
improved user experience, to increased security.  There is strong interest,
primarily outside of academia, in the development of platforms that simplify
cloud-based  application development.

We have prototyped and extensively tested this approach, focusing on a
scalable multiprocess version of our \cb{} platform. The key limitations we
encountered were not aspects of systems architecture or design,  but the
overhead cost of running rich frameworks, primarily in terms of CPU time.   As
these costs shift with the further refinement of these frameworks, we expect
that the use of this approach become more realistic.


%%%%%%%%%%%%%%%%%%%%%%%%%%%%%%%%
\markright{Bibliography}
\bibliographystyle{plain}
\bibliography{../references,../webapps2012,../webappsadditional}
%
% Otherwise, uncomment the following:
% \chapter*{Bibliography}

% \appendix

% In LaTeX, each appendix is a "chapter"
% \chapter{Program Source}


\end{document}
