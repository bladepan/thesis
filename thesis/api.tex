\section{Secure Access to Framework Objects}
\label{sec:api}

Applications frequently need access to our framework's internal objects. 
This is particularly true for \cb's administration interfaces, which are 
themselves implemented as \cb applications.
For instance, in our administration dashboard one can view all the installed
applications and upload new applications. General applications also
require the help of the framework to accomplish common tasks.
For example, in
a chat room application, a user can terminate a chat room (i.e. application
instance) and close all its associated chat windows (i.e. virtual browser),
which are actions that are managed by the application manager objects discussed
in Sections~\ref{sec:masterimpl} and~\ref{sec:worker}.

We designed an API layer through which the applications can call the
framework's methods.  We do not allow applications to call framework
code directly for the following reasons:

\begin{description}

\item[Security] 
% Since a virtual browser's creator is usually the user of
% the virtual browser,
The application code in a given virtual browser runs with
the permissions of the virtual browser's creator.
When the application does not require user identification, then 
the creators of the application's virtual browser are 
set to be the application's owner.

The operations issued by the application code need to conform with the 
permission settings in the \emph{Permission Manager} (discussed in
Section~\ref{sec:worker}).
For example, only the system administrator is
allowed to shutdown all applications. 
These checks take place in the API layer.


\item[Isolation]  Application code must not 
manipulate internal objects in unexpected ways. 

\item[Compatibility]  
We wish to
decouple the application code and the framework code so  that changes made to
the framework code do not break the applications.   

\item[Resource Reclamation]  
Framework objects must be available for garbage collection after they are no
longer needed by the framework,  and the objects created by application code
to be available for garbage collection after their associated virtual browsers
and \appins{}s are closed.  If applications were allowed direct access to the
framework objects,   it would be impossible to fulfill this requirement
since the objects created by application code and framework objects could hold
references to each other.

\end{description}


\subsection{Overall Design}

The \cb API includes four JavaScript classes covering essential features 
applications need to access framework objects.

\begin{description}

\item[APIBrowser] Represents a virtual browser.
It has methods to manipulate the associated virtual browser, including
setting the browser's permissions, closing the browser, etc.

\item[APIAppInstance] Represents an \appins{}. It has methods to manipulate
the associated \appins{} and list the \appins{}'s virtual browsers as 
\emph{APIBrowser} objects.

\item[APIApplication] Represents an application object.
It has methods to manipulate the associated application
and list the application's \appins{}s as \emph{APIAppInstance} objects.

\item[APICloudBrowser]
It has methods to manage applications and list applications as \emph{APIApplication} objects.
For every virtual browser one APICloudBrowser object is created and injected as
a singleton global variable in the application code namespace so that
it can be used by application code.

\end{description}

\apiclassfig{}

As shown in Figure~\ref{fig:apiclass}, based on the initially provided
\emph{APICloudBrowser} object, application code can obtain API objects 
representing applications, application instances and virtual browsers. 
In most applications, the
current application object, the current \appins{} and the current virtual browser are
frequently accessed in application code, thus
the \emph{APICloudBrowser} object provides API object properties to 
these objects directly so that  the application code can access them without 
having to call a chain of methods. Besides framework
objects, the \emph{APICloudBrowser} object also contains properties that
provide auxiliary services such as sending emails.

\subsection{Implementation}

API objects are implemented using a proxy design pattern~\cite{gamma1994design}:  
when their methods are
called, they will first perform permission checking if necessary and then call
the corresponding methods of the framework objects they represent. The proxy
design pattern requires that the proxy objects keep references to the objects they
represent. In languages that feature encapsulation via field modifiers, such 
references are store in private fields so
that the caller of the proxy object cannot obtain a reference to the 
underlying object being represented. Since there is no language support for private
properties in \js, we cannot  store the internal objects as properties of the
API objects because the application code could then obtain a reference to internal
objects using object inspection. 


We use closures to implement an equivalence to private properties~\cite{jsprivate} as
shown in Listing~\ref{code:apiconstructor}. We store internal objects as local
variables in the API objects' constructors and define API methods inside the
constructors. The API methods can reference internal objects
directly because they are part of its closure, even after the constructor has returned.
The application code cannot get references to internal objects because they 
are not stored as properties and they cannot be revealed by reflection.

\begin{listing}[ht,width=\columnwidth]
\begin{minted}[
frame=lines,
fontsize=\scriptsize,
linenos
]
{javascript}
// constructor. virtualBrowser is the actual virtual browser object
function APICloudBrowser(virtualBrowser){
    // store the internal object as a local variable
    var _virtualBrowser = virtualBrowser;
    // define API methods here
    this.close = function(){
        // check if the caller has the privilege to close this virtual browser
        privilegeCheckingCode();
        // forward the operation to the internal object
        _virtualBrowser.close();
    };
}
\end{minted}
\caption{Snippet of Class APIBrowser: the API Class for virtual browsers}
\label{code:apiconstructor}
\end{listing}


% \emph{Need to explain when we would do this. For instance, if an virtual browser
% is closed, then an API object referring to it will be invalid - its weak reference
% will turn to null. Maybe use analogy for OS fds vs OS pids?}

The internal objects' life-cycle should be managed by the framework.
When applications obtain references to the internal objects via the API
layer, they should not prevent the internal objects from being reclaimed.
For instance, when a virtual browser object is referenced indirectly by
an application, the framework should be able to close it and after that
the garbage collector should be able to reclaim the virtual browser object.
To avoid holding references to the internal objects that are no longer needed
by the framework, the API objects only keep weak references to the internal
objects (See Figure~\ref{fig:apireference}). We use a \nodejs{} package node-
weak~\cite{nodeweak} to implement weak reference. Unlike ordinary references,
objects referenced only by weak references are free to be garbage
collected.
In this way, when an internal object is no longer referenced
inside the framework, it is available for garbage collection even 
it is still referenced by the API layer.
After an internal object is garbage collected, the API objects that used to 
represent it are invalid, they
will throw exceptions when their methods are called so that
 the applications will know the internal object ceased to exist.


\apireferencefig{}

The framework may store references to objects created by application
code.  Right now these references are created by the event register methods
provided by the API for the application to register listeners associated with internal
objects. For example, the application can register a event listener to notify
the current user when someone shares an virtual browser with him.  Because these
listeners are registered via the API, the API layer manages the references to these
listeners. When a virtual browser's \code{close} method is called, the API
unregisters all event listeners created by the virtual browser. In this
way, when a virtual browser is closed, the framework won't keep references
to application-created objects that would otherwise cause leaks and prevent
full reclamation.

% \emph{You don't precisely define the security model. If you say a caller's permissions
%     are being checked - really who is the caller?  I think it's virtual browsers
%     are associated with the identity of the authenticated user on whose behalf
%     they were created?  But if user 2 cobrowser a virtual browser instantiated
%     by user 1?  Which identity is in effect when handling an event triggered by
%     user 2? Clarify.}

