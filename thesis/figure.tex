\newcommand{\webscaleoutfig}{
    \begin{figure}[tb]
    \centering
    \includegraphics[width=\textwidth]{../figs/web_scale_out}
    \caption{Scalable web server architecture}
    \label{fig:webscaleout}
    \end{figure}
}

\newcommand{\architectureoverview}{
    \begin{figure*}[ht]
    \centering
    \includegraphics[width=\textwidth]{../figs/architecture_overview}
    \caption{Single Process \cb{} Architecture Overview}
    \label{fig:cb1arch}
    \end{figure*}
}

\newcommand{\newarchitectureoverview}{
    \begin{figure*}[ht]
    \centering
    \includegraphics[width=\textwidth]{../figs/new_architecture_overview}
    \caption{Multiprocess Process \cb{} Architecture Overview}
    \label{fig:cb2arch}
    \end{figure*}
}


\newcommand{\appinstancefig}{
    \begin{figure}[ht]
    \centering
    \includegraphics[width=0.8\textwidth]{../figs/appInstance}
    \caption{Application Instance}
    \label{fig:appinstance}
    \end{figure}
}


\newcommand{\appbundlefig}{
    \begin{figure}[ht]
    \centering
    \includegraphics[width=\textwidth]{../figs/application_bundle}
    \caption{Folder structure of a \cb application}
    \label{fig:appbundle}
    \end{figure}
}

\newcommand{\chatappfig}{
    \begin{figure}[ht]
    \centering
    \includegraphics[width=0.8\textwidth]{../figs/chat_application}
    \caption[Sharing data among multiple virtual browser via application instance]
        {This figure shows how multiple virtual browsers can directly,
        and seamlessly share relevant application data, in this case chat messages,
        which then become part of the model that drives the presentation
        MVC framework.
    }
    \label{fig:chatapp}
    \end{figure}
}

\newcommand{\landingpagefig}{
    \begin{figure}[htb]
    \centering
    \includegraphics[width=\textwidth]{../figs/landing_page}
    \caption[User interface of a landing page]{User interface of a landing page : The user can
    manage all his \appins{}s and virtual browsers including those shared by others in the landing page.}
    \label{fig:landingpage}
    \end{figure}
}


\newcommand{\clickthroughput}{
    \begin{figure}[ht]
    \centering
    \includegraphics[width=\textwidth]{../gnuplot/click_throughput}
    \caption{Throughput of ``Back-to-back'' click application.}
    \label{fig:clickthroughput}
    \end{figure}
}


\newcommand{\clicklatency}{
    \begin{figure}[ht]
    \centering
    \includegraphics[width=\textwidth]{../gnuplot/click_latency}
    \caption{Latency of ``Back-to-back'' click application.}
    \label{fig:clicklatency}
    \end{figure}
}


\newcommand{\clickwaitthroughput}{
    \begin{figure}[ht]
    \centering
    \includegraphics[width=\textwidth]{../gnuplot/click_wait_throughput}
    \caption{Throughput of click application, after introducing artificial delay.}
    \label{fig:clickwaitthroughput}
    \end{figure}
}


\newcommand{\clickwaitlatency}{
    \begin{figure}[tb]
    \centering
    \includegraphics[width=\textwidth]{../gnuplot/click_wait_latency}
    \caption{Latency of click application, after introducing artificial delay.}
    \label{fig:clickwaitlatency}
    \end{figure}
}



\newcommand{\angularchatlatency}{
    \begin{figure}[tb]
    \centering
    \includegraphics[width=\textwidth]{../gnuplot/angularchat_latency}
    \caption{Latency of chat application with Angular.js.}
    \label{fig:angularchatlatency}
    \end{figure}
}


\newcommand{\jquerychatlatency}{
    \begin{figure}[tb]
    \centering
    \includegraphics[width=\textwidth]{../gnuplot/jquerychat_latency}
    \caption{Latency of chat application with JQuery.}
    \label{fig:jquerychatlatency}
    \end{figure}
}


\newcommand{\chatroomfig}{
    \begin{figure}[tb]
    \centering
    \includegraphics[width=\textwidth]{../figs/chatroom}
    \caption{Chat Room Application}
    \label{fig:chatroom}
    \end{figure}
}


\newcommand{\apphierarchyfig}{
    \begin{figure}[tb]
    \centering
    \includegraphics[width=0.8\textwidth]{../figs/application_hierarchy}
    \caption[Application deployment model]{Application deployment model: Hierarchy of applications, application instances, and virtual browsers.
    Note that a single virtual browser may be broadcast to multiple clients (cobrowsing).}
    \label{fig:appidhierarchy}
    \end{figure}
}

\newcommand{\memfig}{
\begin{figure*}[ht]
    \centering
    \includegraphics[width=\textwidth]{../gnuplot/resource_consumption}
    \caption[Resource Consumption of worker]{
    Resource Consumption of worker node running JQueryChat Application\\
X axis is time. Left Y axis corresponds to the red line of CPU usage.
Right Y axis corresponds to memory statistics.\\
After about 90s after the system boots up, the benchmark tool starts to simulate 
user workload.
When the benchmark tool sending requests, \emph{HeapUsed} fluctuates as the system creates new objects and garbage collector cleans dead objects.
When the \emph{HeapUsed} drops there is a steep surge of CPU usage, indicating garbage collector is working at that time.
    }
    \label{fig:mem}
\end{figure*}
}


\newcommand{\nodermifig}{
    \begin{figure}[htb]
    \centering
    \includegraphics[width=0.8\textwidth]{../figs/nodermi}
    \caption[Overall Design of nodermi]{Overall Design of nodermi}
    \label{fig:nodermi}
    \end{figure}
}

% deprecated
\newcommand{\nodermimethodinvokefig}{
    \begin{figure}[htb]
    \centering
    \includegraphics[width=0.8\textwidth]{../figs/nodermi_method_invoke}
    \caption[Remote method invocation via a stub object]{Process A invoke a method of a stub object}
    \label{fig:nodermimethodinvoke}
    \end{figure}
}

% deprecated
\newcommand{\nodermicallbackfig}{
    \begin{figure}[htb]
    \centering
    \includegraphics[width=0.8\textwidth]{../figs/nodermi_callback}
    \caption[Remote callback via a stub function]{Process B invoke a callback that itself is a stub}
    \label{fig:nodermicallback}
    \end{figure}
}

\newcommand{\nodermiexamplefig}{
    \begin{figure}[htb]
    \centering
    \includegraphics[width=0.8\textwidth]{../figs/nodermi_example}
    \caption{Nodermi remote method invocation example}
    \label{fig:nodermiexample}
    \end{figure}
}

\newcommand{\nodermiobjmapfig}{
    \begin{figure}[htb]
    \centering
    \includegraphics[width=0.8\textwidth]{../figs/nodermi_objectmap}
    \caption[Nodermi object map]{Nodermi memory management : Nodermi holds 
    strong references to local objects that are remotely referenced in \emph{object map}.}
    \label{fig:nodermiobjmap}
    \end{figure}
}


\newcommand{\nodermistubmapfig}{
    \begin{figure}[htb]
    \centering
    \includegraphics[width=0.8\textwidth]{../figs/nodermi_objectmap}
    \caption[Nodermi stub map]{Nodermi memory management : Nodermi creates weak
    references to the stubs so it can be notified when stubs are no longer used in
    the system.}
    \label{fig:nodermistubmap}
    \end{figure}
}

%deprecated
\newcommand{\nodermiracefig}{
    \begin{figure}[htb]
    \centering
    \includegraphics[width=0.8\textwidth]{../figs/nodermi_race}
    \caption[Race condition when dereferencing a remote reference]
    {Race condition when dereferencing a remote reference, \emph{objB} is garbage collected
    when \emph{Process A} still has a stub referencing it.}
    \label{fig:nodermirace}
    \end{figure}
}

\newcommand{\nodermipassbyreffig}{
    \begin{figure}[htb]
    \centering
    \includegraphics[width=0.6\textwidth]{../figs/nodermi_passlocalbyreference}
    \caption[Passing by reference in remote method invocations]
    {Passing by reference in remote method invocations:
    When an object is passed by reference to a remote object, the
    receiving process will create a stub to represent that object.}
    \label{fig:nodermipassbyref}
    \end{figure}
}


\newcommand{\nodermipassstubbyreffig}{
    \begin{figure}[htb]
    \centering
    \includegraphics[width=0.6\textwidth]{../figs/nodermi_passstubbyreference}
    \caption[Passing stubs by reference in remote method invocations]
    {Passing stubs by reference in remote method invocations:
    When a stub is passed by reference to another process,
    nodermi creates
    a new stub to represent the stub's source object instead the stub itself.}
    \label{fig:nodermipassstubbyref}
    \end{figure}
}

\newcommand{\nodrmipassbyvalfig}{
    \begin{figure}[htb]
    \centering
    \includegraphics[width=0.6\textwidth]{../figs/nodermi_passbyvalue}
    \caption[Passing arguments by value in remote method invocations]
    {Passing arguments by value in remote method invocations:  
    When \code{objectA} is passed to a remote method, the receiving process creates an exact copy of 
    \code{objectA}.}
    \label{fig:nodermipassbyval}
    \end{figure}
}

\newcommand{\nodrmicyclefig}{
    \begin{figure}[htb]
    \centering
    \includegraphics[width=0.8\textwidth]{../figs/nodermi_cycle}
    \caption[Distributed cyclic reference]
    {Distributed cyclic reference: \code{objA} references \code{stubB}, \code{stubB}
    is a remote reference of \code{objB}, \code{objB} references \code{stubA} which
    is a remote reference of \code{objA}. Neither of them can be garbage collected. }
    \label{fig:nodermicycle}
    \end{figure}
}

\newcommand{\apiclassfig}{
    \begin{figure}[htb]
    \centering
    \includegraphics[width=0.8\textwidth]{../figs/api_classes}
    \caption[API class design]{API class design: 
    The arrow points to the method's return value's type.
    For instance,
    The \emph{listApplications} returns
    a list of \emph{APIApplication} objects.
    ``:'' indicates an variable's class, ``currentApplication:APIApplication'' means
     \emph{currentApplication} is a \emph{APIApplication} object.
    }
    \label{fig:apiclass}
    \end{figure}
}



% http://www.tablesgenerator.com/latex_tables
% supports load table in latex code
\newcommand{\instantiationStrategyTbl}{
    \begin{table}[htb] \footnotesize
    \centering

    \begin{tabular}{|c|l|l|l|l|}
    \hline
    Instantiation Strategy & \begin{tabular}[c]{@{}l@{}}Number of \\ App Instances\end{tabular} & \begin{tabular}[c]{@{}l@{}}Number of Browsers \\ Per App Instance\end{tabular} & \begin{tabular}[c]{@{}l@{}}User Manages \\ App Instances\end{tabular} & \begin{tabular}[c]{@{}l@{}}User Manages\\ Virtual Browser\end{tabular} \\ \hline
    multiInstance & any & any & Yes & Yes \\ \hline
    singleBrowserPerUser & any & 1 per user & Yes & No \\ \hline
    singleInstancePerUser & 1 per user & 1 total & No & No \\ \hline
    singleAppInstance & 1 total & 1 total & No & No \\ \hline
    \end{tabular}


    \caption[Application Instantiation Strategy]{Application Instantiation Strategy}
    \label{tab:appinstantiationstrategy}
    \end{table}
}