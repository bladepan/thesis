\chapter{Background}
\markright{Background}

This chapter gives background information to understand the concept of \cb.
We assume the reader has the basic knowledge of web development.
The reader should understand the meaning of HTTP, HTML, \js and CSS.


\section{Web Cluster}


A web cluster is a group of servers that  
A web server is a instance of software that processes HTTP 


\section{\nodejs}

\cb is written in \nodejs.
\nodejs~\cite{tilkov2010node} is a platform that built on Google Chrome's V8~\cite{v8} \js engine.
Simply put, \nodejs is a standalone \js runtime that allows \js code to be
executed without a web browser.
It is also regarded as server side \js in contrast to typical \js code that executed in 
web browsers.
\nodejs comes with a standard library that provides a rich set of API 
such as file system access, network IO, binary data manipulation, etc. 
that makes it possible to write server side network applications using \js.
\nodejs uses a non-blocking event driven IO model that makes it appealing to
build high performance scalable web applications.
\nodejs also has abundant third party packages
and a great module management system that eases the effort to organize and reuse code.
Another bonus for using \nodejs is that the \cb framework code and application code
is written in the same language, there is language crossing cost.


\begin{listing}[ht,width=\columnwidth]
\begin{minted}[
frame=lines,
fontsize=\scriptsize,
linenos
]
{javascript}
var fs = require('fs');
fs.readFile('/etc/passwd', function (err, data) {
  if (err) throw err;
  console.log(data);
});
console.log("Reading file");
\end{minted}
\caption{Reading file and printing the content on console using \nodejs.}
\label{code:nodefile}
\end{listing}

Listing~\ref{code:nodefile} shows the non-block IO concept in \nodejs.
Because the IO API such as \emph{readFile} does not
block, line 6 will be executed before line 3.
To execute code after the IO operation completes,
the developer needs to register a callback function to be fired asynchronously.
In this example, after the file's content is read the callback function is placed on the
event loop,
then the runtime would dequeue this callback sometime in the future and execute it.
Like in a web browser, \nodejs execute \js code in a single thread,
and the runtime will not halt the current execution in the middle to execute something in
the event loop, so it is easy to write concurrent safe programs.
The event based non-blocking IO and simple execution semantics 
is key to the performance of \nodejs, 
the downside is that any operation that waits for IO must be implemented as a callback.
