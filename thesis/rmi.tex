\chapter{Nodermi: A Remote Procedure Call Framework for \nodejs{}}
\label{ch:rmi}
\markright{nodermi}
In the single process version of \cb,
everything lives in the same address space,
which allows application code to invoke system components' functionality.
As \cbtwo divides \cb into multiple processes,
some of those local method invocations need to be replaced by
inter-process communication,
because as the framework spreads its state
and responsibility to multiple processes,
an operation initiated in one process
could require access to state located in another process.
For example,
in the single process version,
closing a virtual browser
is implemented by invoking the \code{close} method of the virtual browser
object.
If the virtual browser is in another process, closing it requires that a request be sent to that process.
We could have introduced new objects and methods to send specific messages,
but this approach is undesirable for the following reasons.
First, there are many call sites that would need to be changed.
Second, we would also have to design message formats for many different methods and
write handler code to parse and process each of these messages.
Third, since our codebase is continuously evolving, any changes to existing methods
would require modifications to the message communication layer.

To avoid having to manually create and maintain messaging code for each
method, we developed nodermi~\cite{nodermi}, an object-oriented
remote procedure call (RPC) framework for \nodejs{}.
% Nodermi hides the complexity of inter-process communication details from
% upper layer code and is adaptive to code changes.

Before we discuss the implementation of nodermi,
we first introduce some terms for nodermi and
RPC systems in general.
We use the term \emph{local object}
to refer to objects that are allocated by the calling process.
We use the term \emph{remote object} to represent objects that are
located in processes other than the calling process.
We use \emph{stub} or \emph{remote reference} to refer to special objects created
by nodermi that represents \emph{remote object}s in the calling process
(we can think of \emph{stub}s as proxies to remote objects).
We say an object is \emph{remotely referenced} from a process
if there are \emph{stub}s representing it.
Each \emph{stub} represents one \emph{remote object} and
it has methods that mirror the \emph{remote object}'s methods.
From the programmer's perspective,
\emph{stub}s appear like ordinary local objects.
When calling a \emph{stub}'s method, the underlying
RPC framework code sends messages to the \emph{remote object}'s process
to invoke the corresponding method there.
For a given stub,
the object it represents is called its \emph{source object},
the process that creates the \emph{source object}
is referred to as the \emph{host} of the stub.
The \emph{host} process takes on the role of a \emph{server},
whereas the process that holds a \emph{stub} to the object act as a \emph{client}.
In \js{}, since functions are first class citizens, it is possible
to pass a function as an argument or return a function as a result.
Nodermi treats functions and objects alike so that functions can also be
referred to remotely.

% -------- TODO mention the example

\section{Semantics}
\label{sec:semantics}

\nodermiexamplefig{}

\paragraph{Synchronous Methods}
Unlike RPC frameworks in which a calling process blocks during the
completion of the remote method~\cite{birrell1984implementing},
nodermi is fully asynchronous since \js{} functions must not block.
In nodermi a remote method call returns immediately after initiating the communication,
similar to other IO-related asynchronous methods.
Thus the method's return value is not available when the stub method returns on the client.
To obtain the result of a remote method invocation,
the remote method must be implemented asynchronously,
passing its result(s) to the caller via a callback function.
If a caller requires the knowledge that a method has completed - with or without
returning a result, the method must be written in an asynchronous style.
Thus, some existing methods will need to be changed to be able to invoke them
remotely via nodermi.






\paragraph{Passing Arguments}
Nodermi copies most arguments by value, including built-in primitive types
and objects.  If an object refers to other objects, these objects are copied
as well, so that an object's entire transitive closure is copied.
Since JavaScript functions contain bindings to variables that are part of
their closure, it is not possible to copy them.
Instead, we create stubs for functions and methods before passing them
to the remote object's method.  If a method needs to invoke
methods on objects passed as arguments, then such invocations
will cause the created stubs to initiate a remote method
call to the original object.  The client must create handlers
to receive and process these remote method invocations.
For instance, in Figure~\ref{fig:nodermiexample},
when \emph{Process B} calls the remote method \emph{method1}
in \emph{Process A} with a function argument,
a stub for the argument
\emph{callbackA} is created in \emph{Process A} because
it is not possible to copy function \emph{callbackA} to \emph{Process B}.
The stub to \emph{callbackA} is passed as an argument to
 \emph{method1}.
Then \emph{method1} invokes the stub to \emph{callbackA} as if it were an ordinary
 local function and the actual \emph{callbackA} is invoked in \emph{Process A}.

% -------- TODO check the example above

\nodrmipassbyvalfig{}

For certain built-in types, e.g. \code{Date}, \code{Error}, \code{Buffer},
we copy their contents and recreate them via their constructors.
Thus, method invocations on those objects are local, not remote.
In this case, no stubs are created as emphasized in Figure~\ref{fig:nodermipassbyval}.
Besides built-in types, nodermi also allows programmers to
register customized types to be copied and recreated via constructors.
The objects of these customized types need to implement a method to output
their contents such that their constructor can recreate them.
Unlike Java RMI\cite{j2eedoc}'s remote class loading mechanism, we do
not support a way for the sender of such objects to provide the receiver
with code for its methods.  Rather, we use the receiver's versions of
those classes.

\nodermipassbyreffig{}

\nodermipassstubbyreffig{}

Nodermi allows programmers to pass reference to remote objects,
represented by stubs, transparently as arguments to any method.
If we treated such stubs representing remote references like other
objects, we would create a remote stub referring to the passed stub.
If the called method invokes a method on this stub,
the corresponding remote method call would in turn initiate another
remote method invocation to call the object the original stub represents.
Since the remote object may in turn pass the received stub to other
methods, the resulting chain of remote method invocations could
be arbitrary long.
To avoid this ``zig-zag'' problem, when we pass a stub to a remote method call,
we create a new remote reference referencing
 the stub's \emph{source object} instead of referencing the stub itself
 (See Figure~\ref{fig:nodermipassstubbyref}).
Calling methods on this newly created stub will initiate
communication directly with the remote object's host process.
If the remote object resides in the same process as the
receiver object, we pass a direct reference to the object
instead.

\paragraph{Stub Generation.}
A stub's properties are a snapshot of the properties
of its source object at the time the stub is created.
Any subsequent changes to the stub's properties will not
be synchronized with its source object.
Vice versa, any changes to the properties on the source object
have no effect on the stub either.

A stronger consistency  model would propagate changes from
the stubs to the original objects and vice versa, as in DRuby\cite{druby}.
However, it is not feasible to implement this semantics because
in \js{} reading or assigning a object property is a synchronous
operation, thus we cannot engage in any communication to achieve
such synchronization.

An alternative design would have been to disallow properties that
are not methods.
However, in this design, we would need to implement an asynchronous method
for each read property of each object, requiring changes to many
call sites.
This would severely change the structure of existing code
and introduce performance penalties from extra round trips.
For most use cases in our system, providing a snapshot of a
remote object does not affect the correctness of the program.


\section{Design}

\nodermifig{}

As shown in Figure~\ref{fig:nodermi}, a stub's methods call the
\emph{remote method invocation} module in nodermi.
This module first encodes the arguments of the method call
 and other necessary information
in a \emph{method invocation message}.
The transport layer then converts the message to
a binary string and sends it
to the server process hosting the remote object.
On the server side, the transport layer reads
the message from the network and invokes a generic
\emph{method invocation handler}.
The handler reads the description of the arguments
by decoding the \emph{method invocation message}.
Then it recreates the arguments according to
the semantics of argument passing (see Section~\ref{sec:semantics}).
The handler uses the object's id to lookup the corresponding local object.
Finally, the handler invokes the corresponding local method
with the arguments it constructed earlier.

To locate the host process and source object of a stub,
nodermi stores the host process's identifier and the source object's object id
inside the stub so this information can be retrieved later by
the \emph{remote method invocation} module.
The process identifier is a hostname/port pair that each process
allocates for nodermi to listen for incoming TCP messages.
The object id is a unique id nodermi assigns for each object that
is remotely referenced.

Let us explain the message flow with the example shown in Figure~\ref{fig:nodermiexample}.
As shown in the code snippet in the top left of the figure,
process A invokes \emph{method1} with a function argument \emph{callbackA}.
nodermi creates a \emph{method invocation message}
containing the object id of \emph{objB} and the method being called.
The message also contains a description of the arguments:
the object id of \emph{callbackA} and
the type of \emph{callbackA}.
After process B receives the message,
it reconstructs the arguments for the method call:
creating stub \emph{stubCallbackA} for \emph{callbackA} since it is a function.
The stub \emph{stubCallbackA} is a function and
it stores \emph{callbackA}'s id and Process A's process identifier.
After process B finds \emph{objB} via its object id,
process B invokes \emph{method1} with \emph{stubCallbackA} as argument.
\emph{method1} then invokes \emph{stubCallbackA} with \emph{result},
which in turn initiates a remote method invocation
to call \emph{callbackA} in process A.


\paragraph{Bootstrap}
nodermi automatically creates stubs while passing arguments during remote method invocations
on existing stubs, without requiring explicit calls to the framework API.
To bootstrap the RPC communication and obtain an initial stub,
nodermi provides a \code{registerObj} method to register a local object
under a name and a \code{retrieveObj} method to look up a remote object that
has been registered via \code{registerObj} and create a stub for it.

\paragraph{Object Encoding}
When passing an object from one process to another,
nodermi adopts several policies to control the encoding.
Nodermi assumes properties whose names starts with ``\_'' are
private properties and skips them during encoding;
Nodermi skips properties that refer to certain types of objects, e.g. Sockets,
which refer to native OS objects, which cannot be sent.
Nodermi also provides a mechanism for programmers to specify which
additional properties should be ignored during encoding.
Finally, nodermi uses Protobuf~\cite{protobuf}, a very compact
binary format, to serialize its messages.


\section{Distributed Garbage Collection}
\js{} relies on garbage collection to reclaim memory taken by
objects that can no longer be referenced by the program.
However, the garbage collector cannot decide if a local object
could still be referenced by another process via nodermi.
In the example shown in Figure~\ref{fig:nodermiexample},
before \code{method1} of \code{stubB} is called,
the caller creates a stub for \code{callbackA} to be able
to dispatch invocations of this callback.
This stub registers the callback, using its id, in the
method invocation handler's object map.

If nodermi naively kept references to objects such as
\code{callbackA} indefinitely, it would incur memory leaks as all
objects that are even once referenced remotely will never be garbage
collected even after the local and remote processes no longer use them.

Setting timeouts for the references of these objects and cleaning
them automatically is not going to work either,
since there is no guarantee as to when a remote processes will
no longer attempt to use a particular object.
For example, when a process registers a listener function to a remote event publisher,
the listener could be remotely triggered at anytime in the future.

Instead, we implemented a distributed garbage collection mechanism
that ensure that the entry to callbackA is deleted when the
garbage collector is certain that it may no longer be invoked.
Our mechanism is similar to the sequence reference counting algorithm
designed by Birrell et al.~\cite{birrell1993distributed}.

% There are a lot of research related to garbage collection in a distributed
% environment~\cite{abdullahi1998garbage}, ~\cite{birrell1993distributed}.
% We designed a mechanism that is similar to the sequence reference counting algorithm
% designed by Birrell et al.~\cite{birrell1993distributed}.
% We will discuss the difference between our work and ~\cite{birrell1993distributed}
% in section~\ref{sec:relatedrpc}.

The high level design of our distributed garbage collection
algorithm is as follows.
For each process, nodermi maintains an object map containing the objects
that may still be remotely referenced.
The object map prevents the local garbage collector from prematurely garbage collecting
these objects.
This map is also necessary for looking up local objects
when handling \emph{method invocation} messages.

Entries in the object map are removed when a nodermi process learns that there
are no more remote stub referencing the object.   In turn, a nodermi
process must inform the hosts of the objects for it has created stubs
when those stubs become unreachable and subject to garbage collection.

Since \js does not provide finalizers, we use weak
references (implemented using node-weak~\cite{nodeweak}).
Creation of a weak reference to a stub allows us to obtain notification
via a callback when the garbage collector has freed the object.  Weak
references, unlike regular references do not prevent the garbage collector
from doing so.

Thus, when a stub is no longer referenced by application code, it will
be garbage collected and the corresponding weak reference's callback
will execute.  In this callback, nodermi
sends the object's host process a \emph{release message}.

Upon receiving a release message, nodermi knows that a remote
reference to a local object no longer exists and will remove the
corresponding reference from its object map.
An object is garbage collected if the last entry to it from the
object map is removed and there are no other local references to the
object.

%
% Point out that no refcount is done, each peer has its own
% entry.
%

\nodermiobjmapfig{}
\nodermistubmapfig{}

Figure~\ref{fig:nodermiobjmap} shows the structure of the object
map where nodermi keeps the remotely referenced objects
alive for other processes.
Figure~\ref{fig:nodermistubmap} shows the stub map
in which nodermi stores weak references to stubs.
The object map contains the local objects and information about their
remote references.  A remote reference is represented by a session id, a
unique id generated by nodermi every time it transmits objects to remote
processes.  When a local object is transmitted, the current session id
is added to the object map indicating a new remote reference was created.
On the receiving side, the session id is written to the \emph{stub}s
created in this transmission.  When a \emph{stub} is garbage collected,
the \emph{release message} will contain the remote object's id and
the stub's session id.

When a process passes a \emph{stub} to another process and the \emph{stub}
is not from the receiving process, then a new remote reference must be created
to the stub's source object.  In this case, we need to add a record to
the stub's host's object map to represent this new remote reference.
To implement this, before passing the \emph{stub} the sender first
sends a \emph{reference message} to ask the stub's host to generate a
new session id and put that session id into the host's object map.  After
that, the sender passes the \emph{stub} along with the new session id to
the receiver.  The receiver then creates a new \emph{stub} representing
the original object and the new \emph{stub} contains the new session id.

A process could terminate without sending releases messages to release
its remote references.  This leaves records of defunct remote references
in other processes' object maps, thus creating potential memory leaks by
holding up objects that are no longer remotely referenced.  To remove
these dead remote references, nodermi cleans a remote process's records
in the object map if it detects that the remote process has terminated.
To detect termination, nodermi uses a heartbeat mechanism that pings
a process every 60 seconds, unless there is activity from that process.
Nodermi keeps a table with the last response time of each remote process,
which is updated every time it receives a message from a
remote process.  When the table entry for a process is not updated for
more than 60 seconds, nodermi will send a ping message to that process.
If the ping message is not answered within some timeout, all records of that
process will be removed from the object map.  Thus, a terminated process
will be detected and all its remote references will be removed from the system.

\nodrmicyclefig{}

% TBA: discuss cycles. -------- a fiture here example
It is possible for remote references to form distributed cyclic references as shown
in Figure~\ref{fig:nodermicycle}.
Nodermi does not support garbage collect distributed cyclic references.
We think distributed cyclic references are rare in practice and the
programmers can always break the cycle by nullify the properties
introducing the cycles.

% FIXME comparison

% semantics of properties, methods, this  field accesses, object identity
% caja, druby, javascript , python

% \section{Issues}

% async
% network partition, transmission failure, server restart
