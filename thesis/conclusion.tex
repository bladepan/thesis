\chapter{Conclusion}
\markright{Conclusion}

\section{Future Work}

We plan to continue optimize our system by incorporating future updates of
\js language, \js runtime engine and third party dependencies.
The upcoming \js standard brings a lot of promising features.  One of the new
language feature in ECMAScript 7 proposal is native support for observing
changes to objects~\cite{jsobserveprop}. A evaluation shows that dirty
checking in AngularJS becomes 20 to 40 times faster using this
feature~\cite{angularjsspeedup}.
This will greatly improve our system's capacity.


% improve dependency packages... 

We also plan to implement failover mechanism for 
 worker processes. 
In current implementation, when a worker process terminates, 
the master still dispatches requests to that worker.
We need to implement a failure detection mechanism to detect failed workers
and a method to migrate the failed workers' \appins{}s
and virtual browsers to living workers.


Finally, as a PaaS platform we need a better administrative interface 
to enable developers to monitor and diagnose their applications.
Currently, we only have very basic information displayed for deployed 
applications.

\section{Summary}

Maintaining the rich presentation state  of a web application server-side has
many potential benefits, ranging from simplified application development,
improved user experience, to increased security.  There is strong interest,
primarily outside of academia, in the development of platforms that simplify
cloud-based  application development.

We have prototyped and extensively tested this approach, focusing on a
scalable multiprocess version of our \cb{} platform. The key limitations we
encountered were not aspects of systems architecture or design,  but the
overhead cost of running rich frameworks, primarily in terms of CPU time.   As
these costs shift with the further refinement of these frameworks, we expect
that the use of this approach become more realistic.
