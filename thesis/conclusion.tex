\chapter{Conclusion}
\markright{Conclusion}

\section{Future Work}

The upcoming \js standard brings a lot of promising features.  One of the new
language feature in ECMAScript 7 proposal is native support for observing
changes to objects~\cite{jsobserveprop}. A evaluation shows that dirty
checking in Angular.js becomes 20 to 40 times faster using this
feature~\cite{angularjsspeedup}. This will increase our system's capacity when
libraries like Angular.js  adopt this new feature. We can also design new APIs
in \appins to provide notifications of changes to share objects based on this
feature. We will investigate new opportunities brought by language updates
like such to further improve our framework.

% 

% The key limitations we encountered were not aspects of systems architecture or design, 
% but the overhead cost of running rich frameworks, primarily in terms of CPU time.



\section{Summary}

Maintaining the rich presentation state  of a web application server-side has
many potential benefits, ranging from simplified application development,
improved user experience, to increased security.  There is strong interest,
primarily outside of academia, in the development of platforms that simplify
cloud-based  application development.

We have prototyped and extensively tested this approach, focusing on a
scalable multiprocess version of our \cb{} platform. The key limitations we
encountered were not aspects of systems architecture or design,  but the
overhead cost of running rich frameworks, primarily in terms of CPU time.   As
these costs shift with the further refinement of these frameworks, we expect
that the use of this approach become more realistic.
