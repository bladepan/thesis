\section{Design and Implementation}
\label{sec:implementation}

Full details of a single process implementation of the \cb architecture
are available in~\cite{mcdaniel2012cloudbrowser}.  We highlight only the
essential design here.  Figure \ref{fig:cb1arch} shows the relationship 
between the client engine running in the user's browser and the virtual browser
running server side.  When the user visits the application, the client engine
code is downloaded and restores the current view of the application by
copying the current state of the server document.  Subsequently, user input 
is captured, forwarded to the server engine inside the virtual browser, 
which then dispatches it to the document.  All application logic runs
in the global scope associated with the virtual browser's window object.
Since the server environment faithfully mimics a real browser, libraries
such as AngularJS can be used unchanged to implement the user interface.
Client and server communicate through a lightweight RPC protocol that is 
layered on top of a bidirectional web socket communication. 
Stylesheets, images, etc. are provided to the client through a resource
proxy.

\subsection{Application Deployment Model}

We designed an application deployment model for \cbtwo{} that allows different
instantiation strategies.
Application programmers can create CloudBrowser applications in the same way in which 
they create the client-side portion of a client-centric application, using low or high level
JavaScript libraries such as jQuery or AngularJS.  A descriptor in the application's
manifest describes their application's required instantiation strategy.
When a user instantiates an application, the allocated application instance object 
represents metadata about this application
instance, such as ownership and access permissions, along with application data that
the application's code can directly access.  Application instances are created manually
and can be seen in an administration panel (which itself is written as a \cb application).
At any given time, an application instance may have zero or more virtual browsers attached 
to it.  Each virtual browser has its own server-side document and its own global scope, 
but all browsers belonging to an application instance have direct, shared access to the 
application data.  If the application object provides appropriate methods, the application 
data can be saved periodically or upon exit and restored upon server restart.

\chatappfig{}
As an example, consider a scenario for a Chat application developed using AngularJS, 
depicted in Figure~\ref{fig:chatapp}.
A system administrator of a \cb deployment would install the application, which give users the
ability to create application instances. To start a chat site, a user would create
an application instance and share its URL with chat participants.  As the participants join
the chat site, a virtual browser is created on demand for each participant, which is connected 
to the application instance (the users can bookmark their virtual browser's URL to later return.)
The shared application instance data in such an application 
consist of the chatroom(s), users and their associated messages.  The advantage of this design 
is that AngularJS's dirty-checking mechanism will reflect updates to the shared instance data 
in each virtual's browsers document automatically, thus ensuring that new message are broadcast 
to each.

% As shown in Figure~\ref{fig:cb2arch}, 
The hierarchy that results from applications, application instances, and virtual browsers is
depicted in Figure~\ref{fig:appidhierarchy}.  This figure shows the general case in which an
application might allow multiple instances, and in which each user can create multiple virtual
browsers.  We also found it useful to support more specialized instantiation modes, which are
related to the user's authentication state.   These include

\begin{enumerate}
\item \emph{singleAppInstance}. The application supports only one instance and single virtual browser.
    All connected clients will share a single server-side document in this singleton - this can be
    used for applications that display data, such as a weather application. These applications will not
    typically react to user input and users do not need to be authenticated.  
    
\item \emph{singleUserInstance}.  This application requires authentication to establish a
    user identity, which we provide through a local database as well as through external OpenID
    authentication.   In this mode, users may not create more than one virtual browser per
    application instance.  When a user accesses the application instance's URL, they will either
    be forwarded to their virtual browser or a virtual browser will be instantiated for them.
    
\item \emph{multiInstance}. 
    Allows users to have multiple, separate virtual browsers connected to an application
    instance. For instance, a user may have to be in two separate chatrooms offered by one chat site.
    In those cases, the user has the largest flexibility, but will need to manage whether
    to join an existing virtual browser or create a new one when they visit the application instance
    - similar to the choice a user may have when deciding whether to navigate to a new site in 
    an existing browser tab or open a new one.
\end{enumerate}
Except for multiInstance applications, the existence of virtual browsers is not exposed to 
end users that merely join existing application instances.

\apphierarchyfig{}
%
%
\newarchitectureoverview{}
%
\subsection{Distributed Implementation}
\label{sec:distribution}
Figure~\ref{fig:cb2arch} shows the distributed design of \cbtwo, which consists of a single
master process, multiple reverse proxies, and multiple worker processes.  The number of workers 
is determined by the number of CPUs available, due to the event-based (nonblocking) implementation
of the node.js platform on which \cb is implemented.
All processes communicate via standard TCP/IP sockets, so they can be located on a shared multiprocessor 
machine or in a cluster.

The master process maintains a table of which application instances have been created and which
worker is responsible for which application instance.  When a new application instance is created,
the master will apply a load balancing algorithm to decide on which worker to place this instance.  
We support two load balancing strategies: first, the master can assign application instances
to workers in a simple round-robin fashion.  However, since application instances may vary
widely in terms of the actual cost they impose on a worker, we also implemented a load-based
scheme in which workers periodically report a measure of current load to the master.
The master will select the worker with the lowest load when placing a new application instance.
We have found the amount of heap memory that is currently in use a good measure of a worker's 
momentary load.  Once the placement of an instance has been decided, all
virtual browsers created as part of that instance must be located on the worker on which the instance
was placed, as emphasized in Figure~\ref{fig:appidhierarchy}.

Multiple reverse proxy processes are bound to the socket that accepts client requests, allowing
the OS to distributed pending client connections in a round-robin fashion.  
When a proxy process accepts a client, it parses the incoming HTTP request to extract the application
id, which is part of the request's URL.  Each proxy process maintains a table of known mappings from 
application ids to workers.  If the requested id is already cached in that table, the request is directly
relayed to the worker.  Otherwise, the master is contacted to find out which worker is assigned
to that application id, triggering an initial assignment if necessary.  The reverse proxy can 
relay both HTTP requests/responses as well as the bidirectional web socket protocol after the
connection has been upgraded.  Once established, the majority of traffic will be web socket
messages for which there is relatively little per-message overhead.  We implemented the proxy 
using the \nodejs{} module note-http-proxy~\cite{nodeproxy}.
Lastly, worker processes are responsible for maintaining application instances as well as the
creation and management of virtual browser instances.

% \requestdispatchdiagram{}
% \appinstancefig{}

% MARK
\subsection{Interprocess Communication}
Masters, proxies, and workers need to be able communicate with one another. 
This communication takes places at the level of JavaScript method calls.
To facilitate the invocation of JavaScript methods that belong to objects residing 
in other processes, we developed \nodermi, a remote method invocation (RMI) 
library~\cite{nodermi}.

\nodermi uses JavaScript's reflection facilities to transparently create client-side 
stubs for remote objects on the fly.
Since JavaScript does not use static typing and heavily uses variadic functions,
it serializes method arguments dynamically via reflection.  Our system supports all
primitive JavaScript types, objects, and arrays; it can handle cyclic 
object graphs.  It also recognizes when arguments refer to remote objects, in which case 
the receiving side will create its own stub that directly refers to the remote object.
If an argument is a function, a stub will be created on the remote side that, when 
invoked, will make a RPC request back to the requestor.

As a consequence of using RMI, most of the management code in \cbtwo, including the
code implementing the \cb API that is used by the authentication and management applications,
required few changes from their single process version.  A notable exception is that 
methods that return results must do so asynchronously by providing a callback that
is invoked when the result has been obtained.  Changing synchronous method invocations
to asynchronous ones introduce more complex control flow, even in already heavily
asynchronous applications.

