\section{Evaluation}
\label{sec:eval}

% \cite{mcdaniel2012cloudbrowser}
Our main focus in this evaluation is to demonstrate that the multiprocess implementation
in \cbtwo can scale with the number of available cores on which to place separate workers.
Our secondary focus is to investigate the cost of using different client libraries on 
the throughput we achieve.

To test \cbtwo with existing applications, we developed a small expect-like language 
in which to describe scenarios. A client test tool interprets test scripts written in
this language and makes RPC calls to a server in the same way the client engine would
in actual deployment, then checks whether the server is making the expected RPC
calls back that would reflect requests to the client engine to update the DOM the
user sees.  
However, to minimize CPU consumption, the tool does not maintain a DOM in the same
way a real browser would.

Like a (patient) human user, the script will not send the next event until after the
previous event resulted in the expect DOM update.  
Our test language allows us to express repetition and has the ability to check
the correctness of expected responses.  
To simulate human processing speed, a programmable ``thinking delay'' can be introduced 
between receiving an expected response and sending the next event.

To create test scenario from actual applications,
we run those applications using a real browser and record the client/server interactions
in a log.  We then transform this log manually into a test script that is later run 
by our test tool.  To avoid limiting throughput when the test client becomes CPU bound,
we can start multiple instances of the client.

We used a dual-socket, 8 core Intel Xeon 2.27GHz processor with 38GB of RAM on the machine
on which we run \cbtwo{}'s master, worker, and proxies.  We colocate the first proxy with
the master in the same process.
This system is connected via Gigabit Ethernet to a 8 core AMD
machine with 16GB of RAM on which the test client(s) execute. Both machines run 
Ubuntu 14.04 with a stock Linux 3.13 kernel.  We use \nodejs{} 0.10.33 and the version
of JSDOM 2.0 we customized.  We use the static load balancing strategy discussed in 
Section~\ref{sec:distribution} to ensure an exactly predictable distribution of application
instances onto workers.

%
%

\subsection{Click Application}
Our first benchmark is a simple click application that increments a counter on a page
whenever the user clicks.  It is written without any libraries, making direct use of
the JavaScript DOM API.  It only has 14 lines of HTML and 5 lines of JavaScript code.
As such it most closely measures the overhead of our
framework only, e.g. the overhead of making RPC calls, serialization and deserialization,
and dispatching events into the server-side DOM, observing any DOM changes occurring as
a result, and relaying those to the client.

We consider two possible scenarios, a fast ``Back-to-back'' click application 
with no thinking times between clicks, and a more ``Human-paced'' click application
in which there is a think time drawn from a uniform distribution in the range of
1 to 2 seconds.  We perform 5 runs for each data point and report the mean.
For all benchmarks, the variance was small. The vast majority of data points incurred
a relative standard error below 5\%, the maximum relative standard error was 11\%.

\clickthroughput{}
\clicklatency{}

We measure throughput in terms of operations per second and average latency.  
Figure~\ref{fig:clickthroughput} shows the throughput of the application for different
numbers of workers. In those cases, all workers become CPU bound, adding more workers
that can use additional cores increases throughput near linearly up to 6 workers. 
In this scenario, a single proxy is sufficient to support up to
8,000 operations per second at which point it becomes CPU bound;
for the 6 and 8 worker cases, we add a second proxy process.
Since the reverse proxy processes run on the same machine as the workers, throughput 
does not increase beyond 6 workers since this machine has only 8 cores.
We carefully monitor the CPU usage on the benchmark client machine, adding new
test driver instances as needed, up to 8.

\clickwaitthroughput{}
\clickwaitlatency{}

For the back-to-back scenario, throughput is limited by the CPU capacity available
to the workers for 100 clients or more.  Throughput stays relatively constant as
the number of client increases, but the observed latency increases, which is
shown in Figure~\ref{fig:clicklatency}.  Note that we consider a latency of more than
100ms unacceptable~\cite{Nielsen1993Usability}, which is why we cut off the y-axis
accordingly.  As such, in this scenario, a single worker may support up to 200 clients,
and 6 workers can handle about 1,200 clients before latency increases to unacceptable
levels. 

When introducing think times to simulate ``human-paced'' clients, we obtain the
throughput and latency shown in Figures~\ref{fig:clickwaitthroughput} 
and~\ref{fig:clickwaitlatency}.  In those cases, a larger number of clients can be
supported, and maximum throughput will not be reached until the number of
clients ramps up.  For each worker configuration, the maximum acceptable latency 
is reached slightly before maximum throughput is reached.  In these experiments,
we do not increase the number of clients further if latency has already reached
unacceptable levels.
We must note that the latency reported here does not include a wide-area network (WAN) delay;
which when taken into account would reduce the bounds on acceptable processing delay.
Nevertheless, we conclude that our method of scaling to multiple processes is effective 
and that the reverse proxy in particular does not become a bottleneck.

% ----------------------------------------------------------------------------------------------
\subsection{Chat Application} 

A key part of the productivity promise for using the server-centric \cb framework
is the ability to reuse high-level libraries such as AngularJS with little or
no changes, so that applications prototyped in AngularJS can be directly
executed in virtual browsers.  These libraries are designed for client-side
use, however.  We prototyped a Chat application to investigate the overhead
of this usage scenario, a screenshot is shown in Figure~\ref{fig:chatapp}.  
The application is only 207 lines of HTML and JS code while providing features such as 
creating chatrooms, joining them, chatting and changing the desired display name.

\chatroomfig{}

We make use of shared application instance data as discussed in Section~\ref{sec:appmodel}
to hold the last 50 chat messages.  Each application instance is visited by 5 simulated users,
each instantiating their own virtual browser.  Each user sends 100 chat messages in our
script, consisting of a sentence of 15-20 characters.  We set the think time to 5-10 seconds
between messages.  We define latency as the time taken to hit enter and when the  message
appears in the chat window.

\angularchatlatency{}

Fig.\ref{fig:angularchatlatency} shows the latency perceived by benchmark tool
under different numbers of clients. While scaling to multiple workers remains effective,
the use of AngularJS imposes a significant cost, reducing the number of concurrent users 
that can be supported with the same hardware.  Most of the time is spent in the AngularJS
framework, which we discovered through profiling.  Optimizations made in successive revisions
of AngularJS heavily impacted our performance; for instance, a single commit to optimize
needless reexecution of so-called filters in AngularJS improved benchmark performance by 20\%.
As such, our numbers provide limited information about the performance envelope of
server-side environments; rather, they provide a snapshot into the current state of
engineering high-level libraries such as AngularJS.  

\jquerychatlatency{}
To eliminate the impact of AngularJS, we also reprototyped the same application with a
lower-level library, jQuery, which increased its size significantly (and made it significantly
less readable).  Avoiding AngularJS's method of dirty checking to identify model changes
roughly doubled performance, as shown in Figure~\ref{fig:jquerychatlatency}.
AngularJS's developers expect an order of magnitude speedup through direct VM 
support via \code{Object.observe()} in future JavaScript VMs~\cite{angularjsspeedup}.

A second example of a disproportionate impact of specific implementation limitations
is the handling of style attributes.  For instance, jQuery's \code{toggle()} method changes an element's
visibility by changing a style attribute.  To obtain the current value, it uses the
\code{getComputedStyle()} function, which triggers the application of CSS selectors.
This is a well-known performance bottleneck~\cite{Meyerovich+:WWW2010} in browser
layout engines and not optimized at all in JSDOM.  A third example we encountered 
is the original implementation of a frequently called function \code{length\-From\-Properties}
which determines the number of children of a DOM node.  Profiling showed 30\% of CPU
spent in this function, which disappeared after an upgrade that substituted a
constant time $O(1)$ implementation.

