\section{Motivation}
\label{sec:moti}

Our motivation is to create a framework that simplifies scalable web application
development by providing good abstraction of client server interaction and 
server side architecture.
Fig.~\ref{fig:webscaleout} shows the architecture of a scalable web system.
The web layer are servers that generate dynamic web content for clients.
We refer these servers as application servers because they host web application
logic.
The servers in the service layer provide services for the web layer
like data persistence, data query, message queue, cache, etc. 
The clients cannot directly access the service layer.
The load balancer is a logical component that 
distributes requests from clients to application servers.
It could be implemented by one or more special purpose hardware like F5's BIG-IP,
or by reverse proxy software (like nginx, apache HTTP server) deployed on general
purpose servers.
The load balancer could also be a conceptual component that does not map to 
any physical machine in the web system.
For example, 
the load balancer could be implemented by Round Robin DNS,
the web system's owner registers public IP addresses of the application servers
to DNS server under the same name,
when clients send name resolution requests for the name,
the DNS server responses those IP addresses to clients in a round robin fashion,
so clients' actual requests will be distributed to multiple application servers.
We will only focus on the dispatch logic of the load balancer without meddling with
implementation details.
The client layer is the UI representation and application code running on the
clients' web browser.
An actual web system could be much more complex than Fig.~\ref{fig:webscaleout},
for example, there could be multiple layers inside the service layer,
there could be web servers that are dedicated to serve static resources.
We omit these details because they are not relevant to our discussion.



\cb{} aims to provide an integrated solution to the client layer, the load balancer 
and the web layer.


We motivates our approach to 
provide a framework that simplifies web application development
with build in support for developing continuous and collaborative web applications.
People use multiple devices to access web applications all the time, 
preserving the view state of the web application and presenting
users synchronized view on all their devices enhances user experience.
They could use web applications seamlessly on all their devices
without the hassle of manually restoring to the state where they left.
Moreover, as native applications adopts continuous features
to synchronize application state and view among multiple devices, 
web application developers should explore the possibility .%FIXME
In the other way around, native application developers could use \cb{} to 
implement continuous native applications based on HTML.

Web application owners constantly need to increase the capacity of web applications due
to the increase of user loads or the applications themselves demand more resource
as they evolve.
The architecture solutions to scale a web system can be categorized into 
scale-up and scale-out~\cite{cardellini2002state}.
Scale-up is to increase the performance of a single server,
it can be achieved by upgrading the server hardware or optimizing software.
Scale-out is to employ more servers and distribute load among all servers.
In this paper, we will only discuss scale-out solutions and scale-up in
a sense that software could leverage the power of multiple processors.
As other solutions such as increase memory capacity, optimizing operating system,
tuning application software is applicable to all web frameworks.
As Fig.~\ref{fig:webscaleout} shows, 
a scale out web system architecture consists a load balancer and multiple
application servers.
The load balancer distribute requests from clients to the application servers,
application servers host business logic and generate content for client requests.
Here the load balancer could be one or multiple special purpose hardware or reverse 
proxy software like nginx or apache HTTP server deployed on general purpose 
machines, it could also be a conceptual part that handles distribution logic:
for example the web system owner could register IP addresses of all the application
servers to DNS server, and rely on the client browser to distribute its requests to
all the IP addresses.

A common strategy for load balancer is to distribute requests in a content-blind manner,
for example, it pick an application server randomly or in a round robin fashion.
The downside of this approach is that 
for every request the application server needs to restore the application state
as it could receive requests from any user.
For example, the application server need to 
fetch the user name from database when the user's request hit it for the first time,
and this db access is duplicated when the user's requests 
are routed to other servers.

To reduce these redundant work, some web system owners adopt
session affinity distribution.
This approach distributes requests from a user to the same application server,
so the application server needs only create application state once.
In this approach, the load balancer is often implemented by reverse proxy software
like nginx.


In \cb{}, a virtual browser keeps the view 
\cb{} is session-aware by design because 
the user's requests are handled by virtual browsers,
a virtual browser contains the application state and view state.
We will discuss how to distribute virtual browsers among multiple servers
 in detail in section~\ref{sec:implementation}.

On top of virtual browser, 
\cb{} has a level of abstraction called \appins{} that is helpful
to build collaborative applications.
An \appins{} is a group of virtual browsers that are guaranteed to be placed
in the same \cb{} process.
The virtual browsers from the same \appins{} share a user defined \appins{} object.
The developer could use a \appins{} object to describe an entity that 
could be accessed by multiple users, 
then create virtual browser under that \appins{} as user's views of this shared entity.
For example, the application could use an \appins{} object as a shared document and 
define various properties and methods on that object.
Whenever a user want to edit that document, the application
creates a virtual browser under that \appins{},
the user specific application state like user name and editor settings 
could be stored in variables inside his virtual browser.
The application could use the \appins{} object directly to render the document for every user, 
and each user can have his own view of the editor.

Current web frameworks do not have such clear construct to define application state.
One option for the developer is to rely on session objects provided by frameworks.
The application developer needs to configure the persistence  of session objects 
and make sure the configuration works with the load distribution scheme.
For example, in J2EE the session objects could be configured to be stored in memory or
high available database~\cite{j2eedoc}; 
PHP developers need to specify a session save handler to store session objects 
in database~\cite{phpdoc}.
By default, these frameworks store session objects in memory or in local file system.
If the developer adopts a content-blind load balancer, 
he should also configure the session objects to use database or other
synchronization mechanism.
If the developer uses a session affinity load balancer,
he should configure the session objects to be stored in local memory.
Another option is to implement his own way of handling application state.
The developer also needs to make sure his implementation works with
his load balancer.

\cb{} does not have the concept of session,

% That means the applications will not scale if not configured properly.
% Second, the load distribution scheme and 
% Even everything is correctly configured, 
% the developer also needs to write boilerplate code to retrieve objects from the
% session object to restore application state for every request.


% Also, \cb{} makes it clear how the work load is distributed.


\webscaleout{}