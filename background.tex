\section{Background}
\label{sec:background}

Before describing the technical aspects of our design and implementation,
we briefly review a bit of history and related frameworks.
With the emergence of AJAX as a technique over ten years ago a shift commenced from traditional,
page-based applications to the single-page applications in use today. 
The need for architectural support became soon apparent.  The approaches 
that were developed largely fall into two groups: client-centric and server-centric.
Client-centric approaches focused on the creation of powerful JavaScript
libraries that execute in the browser, which is the only language supported
in this environment.  At first, these libraries removed the 
need to handle the many browser incompatibilities that used to be common, 
and they provided the ability to write an application's controller logic 
using higher-level abstractions.  This trend culminated in the creation of 
frameworks which provide components (widgets) from which a UI can be
composed.  More recently, frameworks such as AngularJS added the ability
to implement directives (extensions of HTML) which can be implemented using 
declarative templates.  These directives provide, among others, two-way data 
bindings that connect model variables to the attributes of UI elements such 
as text or checkboxes without requiring the programmer to write controller
code.

The resulting architecture is one in which application developers have enough
high-level facilities at their disposal that it has become feasible to execute 
all user interface related code in the browser.   
Examples include a directive compiler and linker, software engineering features 
such as dependency injection, as well as a ``dirty-checking'' mechanisms that 
detect changes in model variables automatically.
Thus at the extreme end of these frameworks, the backend 
(server side) does not run any user interface logic at all.  Client-side code
interacts with the server through HTTP requests sent to entry points that
retrieve or update only model data, usually represented in JSON as an object or array
that can be easily integrated into the client's model and tied to the rendered 
view.  Application developers write server-side code that implements input 
validation and serializes the data that is sent in responses.
This design creates a clear separation that simplifies server-side 
code as well as load balancing and scaling.

\begin{listing}[ht,width=\columnwidth]
\begin{minted}[
frame=lines,
fontsize=\scriptsize,
linenos
]
{html}
<button ng-hide="showit" ng-click="showit = true">
    Show content!</button>
<button ng-show="showit" ng-click="showit = false">
    Hide content!</button>
<div ng-show="showit">Content here...</div>
\end{minted}
\caption{AngularJS to implement a simple hide/show control 
    that shows/hides content.}
\label{code:angularshowhide}
\end{listing}

However, the user interface programmer must decide when to retrieve the data
from the server and when to save/update the server's state.  Any type of ``server-push''
through must be explicitly programmed.  Moreover, such applications
rarely remember presentation state across visits, and when they do, they do so 
using storage facilities (such as HTML5's \code{localstorage}) that is not 
available on other devices on which a user might use the application.
As a concrete example, consider the snippet of AngularJS code shown in
Listing~\ref{code:angularshowhide}. The value of the model variable \code{showit},
which determines the visibility of a given piece of content, is entirely
client-side and is not remembered across visits unless the programmer writes
code that loads its initial value from the server upon page load and then 
propagates any changes to the server.  Writing this code is possible, but can 
severely undermine the elegance of using high-level frameworks like AngularJS.

By contrast, server-centric approaches represent the user interface on
the server, often using component libraries as well.  Originally, a motivation
for server-centric approaches was that they could be implemented in any language,
such as Java, or C\# which has large ecosystems of support libraries.
In a server-centric approach, the application programmer writes only code
that executes on the server, whereas all client-side code is framework-provided
code.  A pioneering framework in this area was ZK~\cite{ChenCheng:book2007}.
ZK keeps the user interface state in a server-side document that contains
low and high level components, which can be described using declarative
markup and are represented as Java objects.
A client engine receives instructions from the server that render each 
component into appropriate DOM elements.  Application code interacts
with server-side components throught event handlers and attributes, 
creating the appearance of a non-distributed application.  This simplified
model, along with having to maintain a single and much smaller codebase,
is tied to significant productivity gains.
Unfortunately, even in this approach, the user interface representation 
is recreated every time a user navigates to a page.  If programmers wish to
preserve UI state across visits, they have to represent it in session state
or in persistent state such as a database, and retrieve it from there every 
time the view is rebuilt.

% Where to include a preemptive discussion?
