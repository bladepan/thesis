\section{Evaluation}
\label{sec:eval}
We have wrote several web applications on top of \cb{} to test its
performance and scalability.
One application is click application,
it increments and prints a counter on HTML whenever the user clicks.
This application represents applications with simple user interactions and small amount of DOM elements.
Other applications are chat applications.
A user can join a chat room to chat with other people,
changes his nickname, send messages, view recent messages in the chat room.
Whenever the number of messages in a chat room reaches 100, the first 50 messages
are discarded.
The chat applications represents more sophisticated applications.
We implemented the chat applications use popular web frameworks Angular and JQuery
to assess how \cb{} performs with real world web technology stack.

\chatroomfig{}

We implemented a benchmark tool to simulate multiple users.
Each simulated user sends HTTP requests the same way as
a web browser does when an actual user is interacting with a web application.
In the benchmarks, each simulated user will wait for its action taking effect before 
sending the next one.
For example, in the click application, the client will wait for the counter
to be refreshed before clicking again.
In the chat application, the client will wait for
the chat message appearing on the chat window before sending another one.
To be efficient, the simulated user does not render the view, 
it just analyzes server's response message to see if its action has been processed by
the server.
For example, after processing a chat message, 
the chat application will create a new DIV element with the chat message content, 
the simulated user will know a chat message is processed after it received
the corresponding DOM update message.
% TBD

The \cb{} system is deployed on a server with 8 Intel 2.27GHz cores and 12G memory.
The benchmark tool is deployed on a separate machine.

\subsection{Click Application}
In this benchmark,
we allocate a separate virtual browser for each simulated user,
the simulated user will click on the page, 
then wait for the counter to be refreshed and click again.

\clickthroughput{}
\clicklatency{}

\clickwaitthroughput{}
\clickwaitlatency{}

\subsection{Chat Applications}
See Fig.\ref{fig:appinstance},
we use \appins{}s to represent application state of chat rooms.
Each \appins{} has the corresponding chat room's model object ChatRoom.
There could be multiple virtual browsers inside an \appins{} to  
serve requests from users.
In the benchmark, 
we will create simulated users to join chat rooms and send chat messages.
Every five simulated users will be grouped to join one chat room.
Each simulated user will create its own virtual browser.
So there would be one \appins{} for every five virtual browsers.
The simulated user will perform the following actions:
\begin{enumerate}
\item Sleeps for 0-20 seconds, this is simulating that the users
join the chat room.

\item Double click the welcome panel to show the user name editing input box.

\item Input a new name in the user name editing input box and hit enter.

\item \label{itm:chatinput} Input a 15-20 character sentence in the chat message input box and hit enter.

\item Sleeps for 5-10 seconds, and repeat step ~\ref{itm:chatinput}.

\end{enumerate}

% In section \ref{sec:angular} and \ref{sec:jquery}, 
% we will discuss the benchmark

\appinstancefig{}



\subsubsection{Angular Chat Application}
\label{sec:angular}

\angularchatlatency{}

\subsubsection{JQuery Chat Application}
\label{sec:jquery}

\jquerychatlatency{}