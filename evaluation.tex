\section{Evaluation}
\label{sec:eval}
We have wrote several web applications on top of \cb{} to test its
performance and scalability.
One application is called click application,
it increments and prints a counter on HTML whenever the user clicks.
This application represents applications with simple user interactions and small amount of DOM elements.
Other applications are chat applications.
A user can join a chat room to chat with other people,
changes his nickname, send messages, view recent messages in the chat room.
Whenever the number of messages in a chat room reaches 100, the first 50 messages
are discarded.
The chat applications represents more sophisticated applications.
We implemented the chat applications use popular web frameworks Angular and JQuery
to assess how \cb{} performs with real world web technology stack.

We implemented a benchmark tool to simulate multiple users.
The benchmark tool will send HTTP requests to mimic the bootstrapping process,
and send web socket message to \cb{} to simulate user interaction with the application.
These messages are identical to those triggered by real user events like mouse click and key stroke.

In the benchmarks, each simulated client will wait for its action taking effect before 
sending the next one.
For example, in the click application, the client will wait for the counter
to be refreshed before clicking again.
In the chat application, the client will wait for
the chat message appearing on the chat window before sending another one.
% TBD

The \cb{} system is deployed on a server with 8 Intel 2.27GHz cores and 12G memory.
The benchmark tool is deployed on a separate machine.

\subsection{Click Application}
In this benchmark,
we allocate a separate virtual browser for each client,
a client will click on the page, then wait for the counter to be refreshed and click again.

\clickthroughput{}
\clicklatency{}

\clickwaitthroughput{}
\clickwaitlatency{}

\subsection{Angular Chat Application}
In this benchmark,
every 5 clients will share a chat room, 
after join the chat room, the client will change its nickname,
after that the client will start send messages.
The client will wait for each action takes effect to start the next one.
For example, the client will wait for his previous message rendered on the chat window before
sending the next one.


\subsection{JQuery Chat Application}

\jquerychatlatency{}