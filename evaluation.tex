\section{Evaluation}
\label{sec:eval}
We have evaluated how \cb{} scales out in multiple worker nodes and compared the performance
with the original implementation.
The metrics we take into account for measuring performance is memory and cpu consumption on the server side, 
the number of clients
and the request latency perceived on client side.
We create seven test environments : 
the first six environments are deployed with the new implementation with 1, 2, 4, 6, 8 and 10 worker nodes 
configured respectively, 
the last environment is deployed with the original implementation.
All the test environments are deployed on a server with 16 Intel 2.27GHz cores and 12G memory.
For each test environment, 
we deploy a test engine on a separate machine to generate test clients.
Each client connection would trigger some DOM event and send some requests mimicking user interaction with a web application.
There are many factors could impact the performance of the system besides the number of client connections,
like the number of virtual browsers, the memory consumed by each virtual browser, 
the frequency of client requests
and resource taken for handling each client requests.
To make things simpler, we create a simple test application that would create a virtual browser for
every two clients, the application consists a few DOM elements for the client to click,
every client would click the DOM elements and wait for response and click again after a predefined wait time.
For each test environment, 
we gradually increase the number of clients until the average latency exceeds 100ms.



\subsection{Scale Out}
% 1, 2, 4, 6, 8, 10 worker nodes; original implementation 


\subsection{}