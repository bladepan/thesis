\section{Evaluation}
\label{sec:eval}
We have wrote several web applications on top of \cb{} to test its
performance and scalability.
One application is click application,
it increments and prints a counter on HTML whenever the user clicks.
This application represents applications with simple user interactions and small amount of DOM elements.
Other applications are chat applications.
A user can join a chat room to chat with other people,
changes his nickname, send messages, view recent messages in the chat room.
Whenever the number of messages in a chat room reaches 100, the first 50 messages
are discarded.
The chat applications represents more sophisticated applications.
We implemented the chat applications use popular web frameworks Angular and JQuery
to assess how \cb{} performs with real world web technology stack.

\chatroomfig{}

We implemented a benchmark tool to simulate multiple users.
Each simulated user sends HTTP requests the same way as
a web browser does when an actual user is interacting with a web application.
In the benchmarks, each simulated user will wait for its action taking effect before 
sending the next one.
For example, in the click application, the client will wait for the counter
to be refreshed before clicking again.
In the chat application, the client will wait for
the chat message appearing on the chat window before sending another one.
To be efficient, the simulated user does not render the view, 
it just analyzes server's response messages to assess 
if its action has been processed by the server.
For example, after processing a chat message, 
the chat application will create a new DIV element with the chat message content
and append that DIV element to the DOM tree, 
the simulated user will know a chat message is processed after it received
a DOM update message containing that chat message's content.
% TBD

The \cb{} system is deployed on a server with 8 Intel 2.27GHz cores and 12G memory.
The benchmark tool is deployed on a separate machine.

\subsection{Click Application}
In this benchmark,
we allocate a separate virtual browser for each simulated user,
the simulated user will click on the page, 
then wait for the counter to be refreshed and click again.
The application is just a 14 lines HTML document with 5 lines of in-line
\js{} code.
The \js{} code registers a \emph{onclick} handler on a DIV element,
the handler increment a \emph{counter} variable and set the DIV's innerHTML as
the counter every time it is invoked.

\clickthroughput{}

Figure~\ref{fig:clickthroughput} shows the throughput of the application 
under different concurrent levels.
Each line represents a \cb{} with different number of worker nodes.
We can see the throughput scales linearly as the number of worker nodes goes up.
For 100 concurrent simulated clients, 
the throughput of \cb{} with 1, 2, 4, 8 workers are 1191, 2368, 4600 and 7584 events
per second.

\clicklatency{}

Figure~\ref{fig:clicklatency} is the latency of the application.
A usability study shows that the users feel instantaneous if the 
the application response time is below 100ms.
With this criteria,
\cb{} with 8 workers can sustain 800 concurrent users with an acceptable latency (96ms).

This test scenario does not model real clients because real clients cannot 
interact with user interface with such high speed.
We modified our benchmark tools to add a think time 
uniformly distributed from 1 second to 2 seconds before each simulated click.
The throughput and latency for this configuration is shown 
in Figure~\ref{fig:clickwaitthroughput} and Figure~\ref{fig:clickwaitlatency} respectively.
\cb{} can support 10,000 concurrent clients with 8 workers with a latency of 52ms.
% FIXME throughput is not linear


\clickwaitthroughput{}
\clickwaitlatency{}

\subsection{Chat Applications}
As in Fig.\ref{fig:appinstance},
we use \appins{}s to maintain application state of chat rooms.
The virtual browsers use the \emph{ChatRoom} objects inside their \appins{}s 
directly to render the chat history window.
The user can request \emph{Application URL} http://example.com/chat
to create a new chat room.
For example, if 
\emph{userA} requests http://example.com/chat,
an \appins{} and a virtual browser will be created.
Let's say the \appins{}'s id is \emph{appins1},
the virtual browser's id is \emph{vb1}.
\emph{appins1} has an \emph{ChatRoom} object that is used to store 
a chat room's application state.
\emph{vb1} represents \emph{userA}'s view of the newly created
chat room.
If another user \emph{userB} wants to join the chat room, 
he needs to request \emph{appins1}'s \emph{\appins{} URL} 
http://example.com/chat/a/appins1.
The system will create a new virtual browser inside \emph{appins1}
as \emph{userB}'s view.


In the benchmark, 
the simulated users will be grouped into groups of five.
At the beginning of the benchmark, 
for every group one user will request for Application URL to make 
\cb{} create an \appins{},
after that, the remaining users will use the \appins{} URL to start
their own session.
Thus, in the benchmark every five simulated user will share a chat room.

Each simulated user will send 300 chat messages.
Each chat message is a sentence of 15-20 characters.
The user would wait for the previous chat message appended on the chat history
panel before sending the next one.
The user would wait for 5-10 seconds before sending each message,
this is simulating the time for a human to think and type a message.
In the beginning to the benchmark, 
we add a 0-10 seconds wait time for each client before they get started.

We measure the time between the moment 
the user hit the enter key to send the message to 
when the message is echoed back as latency of the chat application.


% The simulated user will perform the following actions:
% \begin{enumerate}
% \item Sleeps for 0-20 seconds. This is simulating the users enter the 
% chat room at different times.

% \item Double click the welcome panel to show the user name editing input box.

% \item Input a new name in the user name editing input box and hit enter.

% \item \label{itm:chatinput} Input a 15-20 character sentence in the chat message input box and hit enter.

% \item Sleeps for 5-10 seconds and repeat step ~\ref{itm:chatinput} 
% until the user has sent 300 messages. 
% This is simulating the time for the user to think and type a message.

% \end{enumerate}


% In section \ref{sec:angular} and \ref{sec:jquery}, 
% we will discuss the benchmark

% http://www.ng-newsletter.com/posts/directives.html

\subsubsection{Angular Chat Application}
\label{sec:angular}
Angular.js~\cite{angular} is a \js{} framework that enable the developers to 
use HTML elements to declare dynamic views.
In this application, we also use bootstrap CSS framework for styling.

Fig.\ref{fig:angularchatlatency} shows the latency perceived by benchmark tool
at different workload.
The system can support much fewer concurrent clients than the click application.
With 8 workers, the system can support 700 concurrent users with average latency 
of 86ms.
First of all, it is a much more complex application than the click application,
the system needs more memory and CPU resource to support each user.
Second, every time the user sends a message, the view of other virtual browsers
in the same chat room needs to be updated as well.
Third, Angular.js brings a substantial overhead 
to pay for the price of friendly programming interface: % confirm, compile ~ link
Angular will walk through the messages list to find out newly created
message objects,
then Angular appends template DOM elements for each new message objects,
finally Angular updates the template DOM elements with the real content of the 
message objects.

Some extra effort is required to make Angular work in our system.
First, if the model object is shared in multiple virtual browsers like this application,
we need to notify other virtual browsers to update their view.
A typical Angular application does not need to update the view explicitly because
Angular would detect model object change and update the view automatically 
after every method's invocation(to be precise, these methods should be declared using Angular's API).
In our environment, angular code in one virtual browser does not know the model object
is changed by some methods in another virtual browser.
However, the code for notify and update the view is just 23 lines of code.
Second, angular use an incremented counter to generate ids to identify objects in an array when the 
array is used in a \emph{ng-repeat} loop.
As we have shared objects for multiple virtual browsers,
different objects created in different virtual browsers could be assigned with duplicate 
ids by different Angular instances.
We create an API to assign unique ids to objects to avoid this problem.
The programmer must call this API before put an object into a data structure that is shared
by multiple virtual browsers.
This problem could also be avoid by letting Angular tracking objects by their position in the array,
we do not use this solution for performance considerations.


\angularchatlatency{}

\subsubsection{JQuery Chat Application}
\label{sec:jquery}
In this application, we use JQuery to manipulate DOM elements.
We also use handlebars.js to create DOM elements based on templates
because writing code manually to generate DOM elements based on dynamic data 
is error prone and difficult to maintain.
To remove compilation overhead, 
we compiled templates into \js{} functions and save the compiled code as static files.
We also use moment.js to do date time formatting as there is no built in support 
in JQuery.
Like the previous application, we use bootstrap for styling and keep the same look
and feel as the previous one.
Compared to Angular, JQuery does not have the expressiveness so we need include 
extra libraries and the code is much less elegant. 

% TODO compare effort
The \js{} plus HTML code for Angular Chat Application is 209 lines long,
for JQuery Chat Application is 258 lines long.
Even taking account of 4 lines API support for Angular.js, 
using Angular.js requires less code than JQuery not mentioning the expressiveness.

Fig.\ref{fig:jquerychatlatency} shows latency of JQuery Chat.
The JQuery Chat supports over twice the number of users comparing to Angular Chat.
The system can support 1,800 concurrent users with 99ms latency.

\jquerychatlatency{}


% code space is part of heapTotal http://jayconrod.com/posts/55/a-tour-of-v8-garbage-collection
\subsection{Memory Consumption}
We use \nodejs{} process.memoryUsage() API to measure the memory usage of each process
of the system.
The function will return rss, heapTotal, and heapUsed of the process.
The \emph{rss} is the size of memory held in RAM by the process.
The \emph{heapTotal} is the size of V8's heap, it is further divided into 
multiple spaces for generational garbage collector and JIT compiler.
The \emph{heapUsed} is the size of heap that the program is currently using,
it includes the size of dead objects that is not garbage collected yet.
These values fluctuate over time as the program creating ephemeral objects and
garbage collector reclaims memory from time to time.
We take memory usage at different phases in 3 experiments.
In these experiments, 
we set up a system of one master and one worker,
then use benchmark tool to create load on JQuery Chat Application mentioned earlier.
As shown in Table~\ref{tbl:mem}, we take the memory statistics from Startup, Peak, and Idle phases.
The Startup phase is when the system starts up before the benchmark tool impose workload.
Peak is the point the highest points of the memory statistics in the experiments.
Idle is after the benchmark tool finishes.
To be more accurate, we force the system do a round of garbage collection before taking the 
\emph{StartUp} and \emph{Idle} value.


At start up, the master node takes up 71MB memory.
It takes more space as the number of connected client increases and the number
of \appins{} created in the system.


% Please add the following required packages to your document preamble:
% \usepackage{multirow}
\begin{table}[h]
\begin{tabular}{|l|l|r|r|r|r|r|r|r|r|r|}
\hline
\multicolumn{2}{|l|}{\multirow{2}{*}{Unit MB}} & \multicolumn{3}{c|}{100 Clients} & \multicolumn{3}{c|}{200 Clients} & \multicolumn{3}{c|}{300 Clients} \\ \cline{3-11} 
\multicolumn{2}{|l|}{} & \multicolumn{1}{l|}{rss} & \multicolumn{1}{l|}{HeapTotal} & \multicolumn{1}{l|}{HeapUsed} & \multicolumn{1}{l|}{rss} & \multicolumn{1}{l|}{HeapTotal} & \multicolumn{1}{l|}{HeapUsed} & \multicolumn{1}{l|}{rss} & \multicolumn{1}{l|}{HeapTotal} & \multicolumn{1}{l|}{HeapUsed} \\ \hline
\multirow{3}{*}{Master} & Startup & 100 & 100 & 100 & 100 & 100 & 100 & 100 & 100 & 100 \\ \cline{2-11} 
 & Peak & 100 & 100 & 100 & 100 & 100 & 100 & 100 & 100 & 100 \\ \cline{2-11} 
 & Idle & 100 & 100 & 100 & 100 & 100 & 100 & 100 & 100 & 100 \\ \hline
\multirow{3}{*}{Worker} & Startup & 100 & 100 & 100 & 100 & 100 & 100 & 100 & 100 & 100 \\ \cline{2-11} 
 & Peak & 100 & 100 & 100 & 100 & 100 & 100 & 100 & 100 & 100 \\ \cline{2-11} 
 & Idle & 100 & 100 & 100 & 100 & 100 & 100 & 100 & 100 & 100 \\ \hline
\end{tabular}
\caption{Memory Usage}
\label{tbl:mem}
\end{table}
