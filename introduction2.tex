%
%
\section{Introduction}
\label{sec:intro}

Most newly developed applications that provide a user interface to end users 
are web-based.  Modern browsers provide powerful and expressive user interface 
elements, allowing for rich applications, and the use of a networked
platform simplifies the distribution of these applications.  As a result,
researchers and practitioners alike have devoted a great deal of attention to
how to architect frameworks on this platform, which is characterized by the
use of the stateless HTTP protocol to transfer HTML-based user interface 
descriptions to the client along with JavaScript code, which in turn 
implements interactivity and communication with the application's backend tiers.

In many recently developed frameworks, much of the presentation logic of these applications
executes within the client's browser.  User input triggers events
which result in information being sent to a server and subsequent user interface updates.
Such updates are implemented as modifications to an in-memory tree-based
representation of the UI (the so-called document object model, or DOM), which
is then rendered by the browser's layout and rendering engine so the user can see it.
However, the state of the DOM is ephemeral in this model: when a user visits 
the same application later from the same or another device, or simply reloads the page, the state of the
DOM must be recreated from scratch.  In most existing applications, this 
reconstruction is done in an rudimentary and incomplete way, because
application programmers typically store only application state, and little 
or no presentation state in a manner that persists across visits.
As a result, many web applications do not truly feel like persistent,
``in-cloud'' applications to which a user can connect and disconnect at will.
By contrast, users are accustomed to features such as Apple's Continuity\cite{apple-continuity}
that allows them to switch between devices while preserving not only 
essential data, but enough of the applications' view to create the appearance
of seamlessly picking up from where they left off.

This paper provides an in-depth exploration of an alternative design that keeps the state 
of the HTML document in memory on the server in a way that is persistent across visits.
In this model, presentation state is kept in virtual browsers whose life
cycle is decoupled from the user's connection state.  When a user is connected,
a client engine mirrors the state of the virtual browser in the actual browser which
renders the user interface the user is looking at.  Any events triggered by the
user are sent to the virtual browser, dispatched there, and any updates
are reflected in the client's mirror.  This idea is reminiscent of 
``thin client'' designs used in cloud-based virtual desktop offerings,
but with the key difference that in this proposed design the presentation state
that is kept in a virtual browser is restricted to what can be represented at the
abstract DOM level; no flow layout or rendering is performed by the virtual browser on the 
server.

This model entails additional potential benefits: since only framework code runs in
the client engine, the application code running on the server does not need
to handle any client/server communication and can be written in an event-based
style similar to that used by desktop user interface frameworks.
Since the virtual browser has the same JavaScript execution capabilities as a
standard browser, emerging model-view-controller (MVC) frameworks such as 
AngularJS can be directly used, further simplifying application development.
More side benefits include: a lighter weight client engine that can load faster,
a resulting application that is potentially more secure since no direct access to
application data needs to be exposed, the ability to cobrowse by broadcasting
the virtual browser state to multiple clients.

We first introduced the idea of HTML-based server-centric execution
in~\cite{mcdaniel2012cloudbrowser}.  In \cbtwo, we provide the first
multi-process implementation of the CloudBrowser concept which enables its use on multiprocessor
machines and on clusters. We have also investigated and implemented several features
commonly expected from a framework, such as authentication and administration.
\cbtwo also includes a model for the deployment of applications and introduces the
concept of an application instance, which allows multiple virtual browsers to share
state.  We have implemented a number of sample and benchmark
applications and profiled them to better understand the intrinsic and extrinsic
limitations of this design.

%
%
%
%
%

% move this until you have the desired placement
\architectureoverview{}

